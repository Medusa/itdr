\documentclass[itdr]{subfiles}

\begin{document}

\section{Tabelle Casuali e Ispirazione}
\label{sec:tabelle_casuali_e_ispirazione}

\subsection{Personaggi Casuali}
\index{Personaggi}

%\subsubsection{Punteggi di Abilità}
\index{Punteggi di Abilità}

\begin{dtable}[c|CCCC]
	\textbf{d20} & \textbf{FOR} & \textbf{DES} & \textbf{VOL} & \textbf{Denaro} \\
	1	& 16& 12& 10& 8s \\
	2	& 10& 12& 16& 8s \\
	3	& 12& 10& 15& 8s \\
	4	& 15& 12& 10& 8s \\
	5	& 10& 14& 12& 8s \\
	6	& 12& 10& 14& 8s \\
	7	& 14& 10& 12& 7s \\
	8	& 12& 15& 10& 6s \\
	9	& 10& 11& 13& 8s \\
	10	& 14& 12& 9 & 7s \\
	11	& 11& 14& 10& 7s \\
	12	& 8 & 16& 13& 5s \\
	13	& 11& 13& 9 & 8s \\
	14	& 9 & 11& 14& 7s \\
	15	& 13& 11& 9 & 7s \\
	16	& 11& 9 & 13& 7s \\
	17	& 13& 9 & 11& 6s \\
	18	& 9 & 13& 11& 6s \\
	19	& 13& 9 & 11& 5s \\
	20	& 9 & 11& 13& 5s \\
\end{dtable}

\vspace{-0.2ex}

\subsubsection{Tratti}
\label{qualità_casuali}
\index{Tratti}

Scegli o tira per un Tratto, tira per i PF come appropriato.

\begin{dtable}[cL|cL]
	\textbf{d3,d5} & \textbf{Tratto} & \textbf{d3,d5} & \textbf{Tratto} \\
	1,1 & Assassino		& 2,4 & Guaritore \\
	1,2 & Belva			& 2,5 & Mistico \\
	1,3 & Berserker		& 3,1 & Provetto \\
	1,4 & Blindato		& 3,2 & Rissaiolo \\
	1,5 & Cecchino		& 3,3 & Spaccone \\
	2,1 & Comandante	& 3,4 & Tattico \\
	2,2 & Combattente	& 3,5 & Taumaturgo \\
	2,3 & Duellante		& ~ \\
\end{dtable}

\subparagraph{Mistico:} tira per stabilire a caso Trucchetti (d20) e Incantesimi del 1° Cerchio (\textbf{tabella dei 36 Incantesimi Random} dal \textbf{\fullref{ch:magia}}\safepageref{, pagina }{incantesimi_random}{}). Sceglie uno di questi Incantesimi con Incantesimo Distintivo o tira un d6 per stabilirne uno a caso.

\subparagraph{Provetto:} tira per due aree di Competenza.
\index{Competenza}

\begin{dtable}[cL|cL]
	\textbf{d10} & \textbf{Competenza} & \textbf{d10} & \textbf{Competenza} \\
	1 & accudimento degli animali	& 6 & negoziazione \\
	2 & atletica						& 7 & raggiro \\
	3 & furtività						& 8 & rapidità \\
	4 & furto con scasso			& 9 & seguire tracce \\
	5 & navigazione					& 10& tolleranza \mbox{degli alcolici} \\
\end{dtable}

\subparagraph{Taumaturgo:} tira per un paio di Doni a caso.

\break

\subsubsection{Background}
\index{Background}

Scegli o tira per un Background e per ogni altra tabella casuale richiesta.

\begin{dtable}[cL|cL]
	\textbf{d8} & \textbf{Background} & \textbf{d8} & \textbf{Background} \\
	1 & Cacciatore	& 5 & Nobile \\
	2 & Criminale	& 6 & Operaio \\
	3 & Marinaio	& 7 & Soldato \\
	4 & Menestrello	& 8 & Studioso \\
\end{dtable}

\vfill

\header{Criminale}
\begin{dtable}[cL|cL]
	\textbf{d6} & \textbf{Arnese da Crimin.} & \textbf{d6} & \textbf{Arnese da Crimin.} \\
	1 & carte segnate	& 4 & manganello \mbox{(sfollagente)} \\
	2 & dadi truccati	& 5 & piede di porco \\
	3 & grimaldelli		& 6 & rampino \\
\end{dtable}

\vfill

\header{Menestrello}
\begin{dtable}[cL]
	\textbf{d10} & \textbf{Strumento Musicale} \\
	1 & arpa, lira \\
	2 & cetra, dulcimer \\
	3 & ciaramella, cromorno \\
	4 & cornamusa, zampogna \\
	5 & flauto, ocarina \\
	6 & ghironda \\
	7 & liuto, mandolino \\
	8 & ribeca, viola, violino \\
	9 & scacciapensieri \\
	10& tamburello, tamburo \\
\end{dtable}

\vfill

\header{Operaio}
\begin{dtable}[clL]
	\textbf{d6} & \textbf{Lavoro} & \textbf{Arma e un paio di Attrezzi} \\
	1 & agricoltura	& forcone, falcetto e setaccio \\
	2 & miniera		& zappa, mazzuolo e trapano \\
	3 & giardinaggio 				& falce, accetta e pala \\
	4 & muratura	& martello da fabbro, secchio e spatola \\
	5 & pastorizia				& bast. ferr., forbici e frusta \\
	6 & taglio di legname 		& maglio spacc., cuneo e sega \\
\end{dtable}

\vfill

\header{Soldato}
\begin{dtable}[cLL]
	\textbf{d6} & \textbf{Grado} & \textbf{Arma} \\
	1	& arciere 		& arco lungo \\
	2	& cavaliere		& lancia lunga\\
	3	& moschettiere 	& moschetto \\
	4	& picchiere 		& picca \\
	5	& spadaccino 	& claymore \\
	6	& ufficiale 		& rivoltella \\
\end{dtable}

\header{Studioso}
\begin{dtable}[cL]
	\textbf{d12} & 	\textbf{Studio} \\
	1	& Filologia (folclore e letteratura) \\
	2	& Filologia (lingue morte) \\
	3	& Filologia (lingue straniere) \\
	4	& Scienze Biologiche (erboristeria) \\
	5	& Scienze Biologiche (medicina) \\
	6	& Scienze Biologiche (zoologia) \\
	7	& Scienze Fisiche (astronomia e fisica) \\
	8	& Scienze Fisiche (chimica) \\
	9	& Scienze Fisiche (geoscienza) \\
	10	& Storia (archeologia) \\
	11	& Storia (cultura e religione) \\
	12	& Storia (geografia e politica) \\
\end{dtable}

\break

\subsubsection{Equipaggiamento}
\index{Equipaggiamento}

Scambia il tuo denaro in cambio di tiri per equipaggiamento a caso. Ripeti i tiri in caso di duplicati. Se il tuo Backgrount ti dà un'arma da mischia, tira invece per una a distanza. I Mistici \mbox{sostituiscono} l'armatura leggera con un'arma da mischia da guerra.

\begin{dtable}[lL]
	\textbf{Denaro} & \textbf{Equipaggiamento} \\
	%4s	& simple M, 2$\times$gear, 2s \\
	5s	& arma semplice da mischia, 2$\times$attrezzatura, attrezzo, 2s \\
	6s	& arma semplice da mischia, arma semplice a distanza, 2$\times$attrezzatura, attrezzo, 2s \\
	7s	& arma semplice da mischia, scudo, 1s \\
	8s	& arma semplice da mischia, arma semplice a distanza, compagno animale, 1s \\
	%9s	& simple M, shield, 2$\times$gear, tool, 1s \\
	%10s	& simple M, simple R, pet, 2$\times$gear, tool, 1s \\
	10s	& arma semplice da mischia, scudo, 2$\times$attrezzatura, attrezzo, 2s \\
	%11s	& simple M, shield, pet \\
	12s	& arma semplice da mischia, arma da guerra a distanza, 1s \\
	14s	& arma semplice da mischia, armatura leggera, 2$\times$attrezzatura, attrezzo, 1s \\
	16s	& arma semplice a distanza, armatura leggera, compagno animale \\
	%18s	& simple M, martial R, pet, 2$\times$gear, 1s \\
	%20s	& martial M, shield, 2$\times$gear, tool, 3s \\
	%22s & martial M, simple R, light armour, 1s \\
\end{dtable}
Oltre a questo, hai vestiti semplici, uno zaino, attrezzatura essenziale per accamparsi, sei torce e razioni per tre giorni.

\vfill

%\dimage{characters}{236pt}
%\dimage{characters2}{370pt}
\dimage{characters3}{282pt}

\vspace{4ex}
\break

\index{Armi}

\header{Armi semplici da mischia (1s)}
\begin{dtable}[cL|cL]
	\textbf{d4} & \textbf{Arma} & \textbf{d4} & \textbf{Arma} \\
	1 & bastone ferrato	& 3 & maglio spaccalegna \\
	2 & forcone 		& 4 & martello da fabbro \\
\end{dtable}

\vfill

\header{Armi da guerra da mischia (10s)}
\begin{dtable}[cL|cL]
	\textbf{d8} & \textbf{Arma} & \textbf{d8} & \textbf{Arma} \\
	1 & alabarda		& 5 & martello da guerra \\
	2 & ascia			& 6 & mazza \\
	3 & lancia			& 7 & pugnale \\
	4 & lancia lunga 	& 8 & spada \\
\end{dtable}

\vfill

\header{Armi semplici a distanza (1s)}
\begin{dtable}[cL|cL]
	\textbf{d6} & \textbf{Arma} & \textbf{d6} & \textbf{Arma} \\
	1 & arco da caccia	& 4 & freccette \\
	2 & boomerang		& 5 & pugnali da lancio \\
	3 & fionda			& 6 & shuriken \\
\end{dtable}

\vfill

\header{Armi da guerra a distanza (10s)}
\begin{dtable}[cL|cL]
	\textbf{d4} & \textbf{Arma} & \textbf{d4} & \textbf{Arma} \\
	1 & arco lungo	& 3 & moschetto \\
	2 & balestra		& 4 & rivoltella \\
\end{dtable}

\vfill

\index{Attrezzatura d'Avventura}
\header{Attrezzatura d'Avventura (5p)}
\begin{dtable}[cL|cL]
	\textbf{d12} & \textbf{Attrezzatura} & \textbf{d12} & \textbf{Attrezzatura} \\
	1 & bottiglia				& 7	& esca e pietra focaia \\
	2 & candela				& 8 & gessetto \\
	3 & cartapecora			& 9	& sacco \\
	4 & carte o dadi			& 10& spuntone \\
	5 & catena				& 11& tenda \\
	6 & corda di 10 piedi		& 12& triboli \\
\end{dtable}

\vfill

\index{Attrezzi}
\header{Attrezzi (1s)}
\begin{dtable}[cL|cL]
	\textbf{d20} & \textbf{Tool} & \textbf{d20} & \textbf{Tool} \\
	1 & accetta					& 11& pertica ripieghevole \\
	2 & canna da pesca			& 12& piccozza \\
	3 & chiave serratubi			& 13& piede di porco \\
	4 & forbici					& 14& pinza \\
	5 & grimaldelli				& 15& rampino \\
	6 & lima o raspa			& 16& sega \\
	7 & lucchetto				& 17& set da scrittura \\
	8 &	martello					& 18& tenaglia \\
	9 & mazzuola \mbox{e scalpello}	& 19& trapano \\
	10& pala					& 20& trappola \mbox{per animali} \\
\end{dtable}

\vfill

\index{Compagni Animali}
\header{Compagno Animale (5s)}
\begin{dtable}[cL|cL]
	\textbf{d4} & \textbf{Compagno Anim.} & \textbf{d4} & \textbf{Compagno Anim.} \\
	1 & gatto 	& 3 & meticcio \\
	2 & gufo	& 4 & pappagallo \\
\end{dtable}

\vfill
\break

\subsection{Oggetti Magici Casuali}
\phantomsection
\label{subsec:oggetti_magici_casuali}
\index{Magia!oggetti}
\index{Tesori}

Tira per un oggetto magico a caso e per il suo aspetto esteriore. Pensa alle sue proprietà basandoti su come appare.

\header{Tipo}
\begin{dtable}[cLcl]
	\textbf{d100} & \textbf{Tipo} & \textbf{d100} & \textbf{Tipo} \\
	1--10	&	contenitore		&	71--73		&	strumento musicale	\\
	11--30	&	consumabile	&	74--80		&	armatura leggera	\\
	31--40	&	indumento		&	81--83		&	armatura completa	\\
	41--50	&	gioiello			&	84--90		&	scudo	\\
	51--70	&	miscellanea		&	91--100	&	arma	\\
\end{dtable}

\vspace{0.5ex}
\dimage{magicitems}{80pt}
\vspace{0.5ex}

\header{Contenitore}
\begin{dtable}[cLcl]
	\textbf{d12} & \textbf{Contenitore} & \textbf{d12} & \textbf{Contenitore} \\
	1 & borraccia o fiasca	& 7 & corno potorio \\
	2 & borsa				& 8 & faretra \\
	3 & bottiglia			& 9 & fiala \\
	4 & brocca				& 10& otre \\
	5 & caraffa				& 11& sacchetto o sacco \\
	6 & cofanetto o scatola	& 12& tascapane o zaino \\
\end{dtable}

\vfill

\header{Consumabile}
\begin{dtable}[cLcL]
	\textbf{d10} & \textbf{Consumabile} & \textbf{d10} & \textbf{Consumabile} \\
	1 & \mbox{balsamo, olio} \mbox{o unguento}	& 6 & \mbox{erba, fiore} \mbox{o foglia} \\
	2 & candela o torcia 							& 7 & gessetto o matita \\
	3 & cibo (frutto, impasto, \mbox{ecc.)} 			& 8 & inchiostro \mbox{o pittura} \\
	4 & cipria o polvere								& 9 & \mbox{legume, radice} \mbox{o seme} \\
	5 & elisir o pozione 								& 10& veleno \\ 
\end{dtable}

\vfill

\header{Indumento}
\begin{dtable}[cLcL]
	\textbf{d20} & \textbf{Indumento} & \textbf{d20} & \textbf{Indumento} \\
	1	& abito			& 11 & guanti \\
	2	& brache		& 12 & mantello \\
	3	& camicia		& 13 & manto \\
	4	& cappello		& 14 & pantaloni \\
	5	& cappotto		& 15 & paramenti \\
	6	& cappuccio		& 16 & sandali \\
	7	& cintura		& 17 & scarpe \\
	8	& farsetto		& 18 & stivali \\
	9	& giubba		& 19 & tunica \\
	10	& gonnella		& 20 & veste \\
\end{dtable}

\vfill
\break

~\\
\header{Gioiello}
\begin{dtable}[cLcL]
	\textbf{d20} & \textbf{Gioiello} & \textbf{d20} & \textbf{Gioiello} \\
	1	& anello 				& 11 & fibbia da cintura \\
	2	& benda oculare 		& 12 & forcina \\
	3	& braccialetto 			& 13 & gorgiera \\
	4	& catenina 				& 14 & maschera \\
	5	& cavigliera 			& 15 & medaglietta \\
	6	& collana 				& 16 & medaglione \\
	7	& corona \mbox{o coroncina}	& 17 & orecchino \\
	8	& diadema o tiara 		& 18 & pendente \\
	9	& fascia 				& 19 & pettorale \\
	10	& fermaglio 			& 20 & spilla da mantello \\
\end{dtable}

\vfill

\header{Miscellanea}
\begin{dtable}[cLcl]
	\textbf{d100} & \textbf{Oggetto} & \textbf{d100} & \textbf{Oggetto} \\
	1--2	&	ago								&	51--52		&	lente o monocolo	\\
	3--4	&	amuleto \mbox{o talismano}		&	53--54		&	libro	\\
	5--6	&	bacchetta						&	55--56		&	manette	\\
	7--8	&	bastone							&	57--58		&	martello	\\
	9--10	&	braciere							&	59--60		&	moneta	\\
	11--12	&	calice, coppa \mbox{o coppetta}&	61--62		&	occhiali	\\
	13--14	&	candelabro						&	63--64		&	ombrello	\\
	15--16	&	cannocchiale					&	65--66		&	pala	\\
	17--18	&	carte o dadi						&	67--68		&	penna d'oca	\\
	19--20	&	cavatappi						&	69--70		&	pettine	\\
	21--22	&	ciotola \mbox{o secchio}		&	71--72		&	piatto o vassoio	\\
	23--24	&	clessidra						&	73--74		&	piccozza	\\
	25--26	&	corda							&	75--76		&	pipa	\\
	27--28	&	cristallo o sfera					&	77--78		&	protesi	\\
	29--30	&	falcetto							&	79--80		&	scettro o verga	\\
	31--32	&	fazzoletto						&	81--82		&	scopa	\\
	33--34	&	ferro di cavallo					&	83--84		&	sella	\\
	35--36	&	fischietto						&	85--86		&	spazzola	\\
	37--38	&	forbici							&	87--88		&	specchio	\\
	39--40	&	gancio							&	89--90		&	spuntone	\\
	41--42	&	gemma o perla					&	91--92		&	tappeto	\\
	43--44	&	grimaldello						&	93--94		&	tavoletta	\\
	45--46	&	idolo \mbox{o statuetta}		&	95--96		&	teschio	\\
	47--48	&	incensiere						&	97--98		&	tovaglia	\\
	49--50	&	lanterna							&	99--100	&	ventaglio	\\
\end{dtable}

\vfill
\break

\header{Strumenti Musicali}
\begin{dtable}[cLcL]
	\textbf{d20} & \textbf{Strumento} & \textbf{d20} & \textbf{Strumento} \\
	1	& arpa 			& 11 & liuto \\
	2	& campana 		& 12 & mandolino \\
	3	& cetra			& 13 & ocarina \\
	4	& ciaramella 	& 14 & ribeca \\
	5	& cornamusa 	& 15 & scacciapensieri \\
	6	& cromorno 	& 16 & tamburello \\
	7	& dulcimer 		& 17 & tamburo \\
	8	& flauto 		& 18 & viola \\
	9	& ghironda 		& 19 & violino \\
	10	& lira 			& 20 & zampogna \\
\end{dtable}

\vfill

\header{Armatura Leggera e Accessori}
\begin{dtable}[cLcL]
	\textbf{d6} & \textbf{Armatura} & \textbf{d6} & \textbf{Armatura} \\
	1 & armatura di cuoio		& 4 & gambesone \\
	2 & bracciali	& 5 & guanti \\
	3 & elmo		& 6 & schinieri \\
\end{dtable}

\vfill

\header{Armatura Completa e Accessori}
\begin{dtable}[cLcl]
	\textbf{d10} & \textbf{Armatura} & \textbf{d10} & \textbf{Armatura} \\
	1 & armatura a strisce		& 6 & corazza \\
	2 & armatura di maglia		& 7 & elmo \\
	3 & armatura di piastre	& 8 & gambali \\
	4 & armatura di scaglie		& 9 & manopole \\
	5 & bracciali		& 10& schinieri \\
\end{dtable}

\vfill

\header{Scudo}
\begin{dtable}[cLcL]
	\textbf{d6} & \textbf{Scudo} & \textbf{d6} & \textbf{Scudo} \\
	1 & brocchiero			& 4 & scudo quadrato \\
	2 & palvese	& 5 & scudo rotondo \\
	3 & scudo a mandorla		& 6 & scudo scapezzato \\
\end{dtable}

\vfill

\header{Arma e Munizioni}
\begin{dtable}[cLcL]
	\textbf{d20} & \textbf{Arma} & \textbf{d20} & \textbf{Arma} \\
	1 & alabarda		& 11 & lancia lunga \\
	2 & arco da caccia			& 12 & martello da guerra \\
	3 & arco lungo		& 13 & mazza \\
	4 & ascia	& 14 & moschetto \\
	5 & balestra		& 15 & pallottola \\
	6 & boomerang	& 16 & pugnale \\
	7 & fionda		& 17 & quadrello \\
	8 & freccetta		& 18 & rivoltella \\
	9 & freccia		& 19 & shuriken \\
	10& lancia	& 20 & spada \\
\end{dtable}

\vspace{4ex}
\break

\subsubsection{Aspetto Casuale}

\header{Attributo (quando appropriato)}
\begin{dtable}[cLcL]
	\textbf{d20} & \textbf{Attributo} & \textbf{d20} & \textbf{Attributo} \\
	1	&	ampolloso	&	11	&	leggero o sottile	\\
	2	&	antico	&	12	&	lucente	\\
	3	&	colorato	&	13	&	minaccioso	\\
	4	&	complicato	&	14	&	particolare	\\
	5	&	decorato	&	15	&	pesante	\\
	6	&	elegante	&	16	&	raffinato	\\
	7	&	esotico	&	17	&	sofisticato	\\
	8	&	grezzo	&	18	&	solido	\\
	9	&	grottesco	&	19	&	squallido	\\
	10	&	ingioiellato	&	20	&	ultraterreno	\\
\end{dtable}

\vfill

\header{Colore (quando appropriato)}
Usa la tabella dei colori per i \textbf{\safenameref{sec:contrattempi_magici}{Contrattempi Magici}}\safepageref{ a~pagina }{sec:contrattempi_magici}{}.

\vfill

\header{Materiale (quando appropriato)}
\begin{dtable}[cLcL]
	\textbf{d20} & \textbf{Materiale} & \textbf{d20} & \textbf{Materiale} \\
	1 & acciaio	& 11 & gaietto \\
	2 & ambra	& 12 & giada \\
	3 & argento	& 13 & legno \\
	4 & avorio o corno	& 14 & oro \\
	5 & bronzo	& 15 & ossidiana \\
	6 & ceramica	& 16 & ottone \\
	7 & chitina o osso	& 17 & peltro \\
	8 & corallo	& 18 & pietra \\
	9 & cristallo	& 19 & rame \\
	10& ferro	& 20 & vetro \\
\end{dtable}

\vfill

\header{Peculiarità (1 su 6 possibilità di essere presente)}
\begin{dtable}[cL]
	\textbf{d12} & \textbf{Peculiarità} \\
	1 & cambia colore quando non c'è chi l'osserva \\
	2 & al tocco, trasmette una sensazione di freddo \\
	3 & emette un ronzio appena udibile \\
	4 & brilla leggermente al buio \\
	5 & pesa più di quanto sembri \\
	6 & pesa meno di quanto sembri \\
	7 & al tocco, trasmette una sensazione di melma o olio \\
	8 & è semitrasparente \\
	9 & ha un odore bizzarro, ma non sgradevole \\
	10& a volte pare che si muova lievemente \\
	11& ogni tanto vibra un po' \\
	12& al tocco, trasmette una sensazione di calore \\
\end{dtable}

\vfill

\header{Trama (quando appropriato)}
\begin{dtable}[cLcL]
	\textbf{d8} & \textbf{Trama} & \textbf{d8} & \textbf{Trama} \\
	1 & cotone	& 5 & lino \\
	2 & crine	& 6 & pelle \\
	3 & feltro		& 7 & pelliccia \\
	4 & lana 	& 8	& seta \\
\end{dtable}

\vfill
\break

\subsection{Mostri Casuali}
\index{Mostri}

Per scegliere quale tabella usare, tira il d8 per d4 volte. Ripeti il tiro in caso di duplicati. Poi, usa la tabella \textbf{Forma}.

\vfill

%\header{1. Nature}
\begin{dtable}[cLcL]
	\textbf{d12} & \textbf{1. Natura} & \textbf{d12} & \textbf{1. Natura} \\
	1	&	artificiale	&	7	&	magico	\\
	2	&	coloniale	&	8	&	mutato	\\
	3	&	derelitto	&	9	&	nella norma	\\
	4	&	divino	&	10	&	non morto	\\
	5	&	etereo	&	11	&	occulto	\\
	6	&	infernale	&	12	&	primitivo	\\
\end{dtable}

\vfill

%\header{2. Appearance}
\begin{dtable}[cLcL]
	\textbf{d20} & \textbf{2. Aspetto} & \textbf{d20} & \textbf{2. Aspetto} \\
	1	&	aggraziato	&	11	&	muscoloso	\\
	2	&	ammalato	&	12	&	non è visibile	\\
	3	&	con barbigli	&	13	&	pelato	\\
	4	&	limaccioso	&	14	&	peloso	\\
	5	&	luccicante	&	15	&	putrescente	\\
	6	&	maculato	&	16	&	radioso	\\
	7	&	magro	&	17	&	rigato	\\
	8	&	maleodorante	&	18	&	rigonfio	\\
	9	&	mimetizzato	&	19	&	rugginoso	\\
	10	&	multicolore	&	20	&	tenebroso	\\
\end{dtable}

\vfill

%\header{3. Trait}
\begin{dtable}[cLcL]
	\textbf{d20} & \textbf{3. Tratto} & \textbf{d20} & \textbf{3. Tratto} \\
	1	&	acido	&	11	&	ipnotico	\\
	2	&	acustico	&	12	&	minuscolo	\\
	3	&	adesivo	&	13	&	munito di guscio	\\
	4	&	armato	&	14	&	parassita	\\
	5	&	corazzato	&	15	&	psichico	\\
	6	&	elettrico	&	16	&	rigurgita	\\
	7	&	fuoco	&	17	&	si moltiplica	\\
	8	&	ghiaccio	&	18	&	spara proiettili	\\
	9	&	gigante	&	19	&	vampirico	\\
	10	&	ingoia	&	20	&	velenoso	\\
\end{dtable}

\vfill

%\header{4. Behaviour}
\begin{dtable}[cLcL]
	\textbf{d20} & \textbf{4. Condotta} & \textbf{d20} & \textbf{4. Condotta} \\
	1	&	amichevole	&	11	&	intelligente	\\
	2	&	astuto	&	12	&	musicale	\\
	3	&	avido	&	13	&	notturno	\\
	4	&	brulicante	&	14	&	pacifico	\\
	5	&	divoratore	&	15	&	saprofago	\\
	6	&	elusivo	&	16	&	si avvinghia	\\
	7	&	farfugliante	&	17	&	silenzioso	\\
	8	&	folle	&	18	&	sussurrante	\\
	9	&	furente	&	19	&	tende imboscate	\\
	10	&	imprevedibile	&	20	&	urlante	\\
\end{dtable}

\vfill
\break

%\noindent~\vspace{1.3ex}\\
%\header{5. Locomotion}
\begin{dtable}[cLcL]
	\textbf{d20} & \textbf{5. Locomozione} & \textbf{d20} & \textbf{5. Locomozione} \\
	1	&	acquatico	&	11	&	scava cunicoli	\\
	2	&	cammina	&	12	&	scorre	\\
	3	&	corre	&	13	&	serpeggia	\\
	4	&	dinoccolato	&	14	&	si arrampica	\\
	5	&	immobile	&	15	&	si impenna	\\
	6	&	fluttua	&	16	&	si teletrasporta	\\
	7	&	lento	&	17	&	sotterraneo	\\
	8	&	plana	&	18	&	striscia	\\
	9	&	rotola	&	19	&	veloce	\\
	10	&	salta	&	20	&	vola	\\
\end{dtable}

\vfill

%\header{6. Body}
\begin{dtable}[cLcL]
	\textbf{d20} & \textbf{6. Corpo} & \textbf{d20} & \textbf{6. Corpo} \\
	1	&	alato		&	11	&	privo di un corpo	\\
	2	&	arti multipli	&	12	&	quattro braccia	\\
	3	&	asimmetrico	&	13	&	quattro gambe	\\
	4	&	braccia multiple	&	14	&	senza arti	\\
	5	&	dotato di coda	&	15	&	senza braccia	\\
	6	&	due braccia		&	16	&	senza gambe	\\
	7	&	due gambe	&	17	&	sferico	\\
	8	&	due teste	&	18	&	struttura radiale	\\
	9	&	gambe multiple	&	19	&	un braccio	\\
	10	&	munito \mbox{di tentacoli}	&	20	&	una gamba	\\
\end{dtable}

\vfill

%\header{7. Head}
\begin{dtable}[cLcl]
	\textbf{d12} & \textbf{7. Testa} & \textbf{d12} & \textbf{7. Testa} \\
	1	&	bicefalo	&	7	&	munito di corna	\\
	2	&	cieco	&	8	&	muto	\\
	3	&	con molteplici occhi	&	9	&	senza cervello	\\
	4	&	con molteplici teste	&	10	&	senza occhi	\\
	5	&	con proboscide \mbox{o tentacoli}	&	11	&	senza testa	\\
	6	&	con un occhio solo	&	12	&	sordo	\\
\end{dtable}

\vfill

%\header{8. Material}
\begin{dtable}[clcL]
	\textbf{d10} & \textbf{8. Materiale} & \textbf{d10} & \textbf{8. Materiale} \\
	1 & argilla, fango, liquame		& 6 &	di legno	\\
	2 & chitinoso, osso	& 7 &	elementale, \mbox{gassoso}	\\
	3 & cristallino, gemma		& 8 &	liquido \\
	4 & cuoio, tessuto	& 9 &	metallico	\\
	5 & di carne	& 10&	pietra	\\
\end{dtable}

\vfill

%\header{9. Form}
\begin{dtable}[cLcL]
	\textbf{d20} & \textbf{9. Forma} & \textbf{d20} & \textbf{9. Forma} \\
	1	&	amorfo	&	11	&	pipistrello	\\
	2	&	anfibio	&	12	&	rettile, serpente	\\
	3	&	animato	&	13	&	roditore, coniglio,\newline riccio, talpa,\newline toporagno, ecc.	\\
	4	&	aracnide, insetto	&	14	&	simile a un cane	\\
	5	&	con zoccoli	&	15	&	simile a un gatto	\\
	6	&	crostaceo, miriapode	&	16	&	simile a un orso	\\
	7	&	fungo	&	17	&	ucello	\\
	8	&	mollusco, verme	&	18	&	umanoide	\\
	9	&	pesce	&	19	&	chimerico*	\\
	10	&	pianta	&	20	&	mutaforma*	\\
	\end{dtable}
{\em* Tira altre due volte}

\break

\subsection{Personaggi Non-Giocanti Casuali}
\index{Personaggi Non-Giocanti}

\vfill

\header{Età e Ricchezza}
\begin{dtable}[cL|cL]
	\textbf{d8} & \textbf{Età} & \textbf{d6} & \textbf{Ricchezza} \\
	1--2 & giovane 		& 1--2 & povero \\
	3--6 & di mezza età	& 3--5 & nella media \\
	7--8 & vecchio			& 6	 & ricco \\
\end{dtable}

\vfill

\header{Occupazione}
\begin{dtable}[cLcL]
	\textbf{3d6} & \textbf{Occupazione} & \textbf{3d6} & \textbf{Occupazione} \\
	3 & studioso				& 11 & artigiano \\
	4 & guaritore				& 12 & servitore \\
	5 & artista				& 13 & mercante \\
	6 & intrattenitore			& 14 & guardia, \mbox{soldato} \\
	7 & criminale			& 15 & marinario \\
	8 & mendicante, \mbox{vagabondo}	& 16 & scriba,\newline segretario \\
	9 & cacciatore, \mbox{pescatore}	& 17 & sacerdote \\
	10& contadino, \mbox{paesano}		& 18 & nobile \\
\end{dtable}

\vfill

\header{Personalità}
\begin{dtable}[cLcL]
	\textbf{d20} & \textbf{Personalità} & \textbf{d20} & \textbf{Personalità} \\
	1 & amichevole	& 11 & generoso \\
	2 & arrogante		& 12 & gioioso \\
	3 & avido	& 13 & melanconico \\
	4 & collerico	& 14 & modesto \\
	5 & cortese	& 15 & onesto \\
	6 & credulone	& 16 & ostile \\
	7 & diffidente		& 17 & rude \\
	8 & disattento	& 18 & sveglio \\
	9 & disonesto		& 19 & tonto \\
	10& ficcanaso	& 20 & tranquillo \\
\end{dtable}

\vfill

Tira due volte per i dettagli degni di nota, ripetendo il tiro laddove inappropriato.

\header{Dettaglio Degno di Nota}
\begin{dtable}[cLcl]
	\textbf{3d8} & \textbf{Dettaglio} & \textbf{3d8} & \textbf{Dettaglio} \\
	3 & gobbo	& 14 & alto \\
	4 & con un occhio solo		& 15 & sovrappeso \\
	5 & cicatrice		& 16 & baffi \\
	6 & balbuzie		& 17 & capelli lunghi \\
	7 & alcolizzato	& 18 & basette \\
	8 & capelli grigi	& 19 & capelli di colore raro* \\
	9 & calvo		& 20 & accento \\
	10& capelli corti	& 21 & voglia \\
	11& barba folta & 22 & occhio pigro \\
	12& esile		& 23 & protesi a una gamba \\
	13& basso		& 24 & protesi a un braccio \\
\end{dtable}
{\em* Di solito biondo o rosso, in base alla popolazione generale.}

\vfill
\break

~\vspace{5ex}\\
\dimage{street}{542pt}

\end{document}