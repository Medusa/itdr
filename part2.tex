\documentclass[itdr]{subfiles}

\begin{document}

\cleartoleftpage

\chapter{Giocare}
\label{ch:giocare}

\section{Regole Basilari}
\index{Regole Basilari}
\index{Arbitro}

\paragraph{Tiri Salvezza}
\index{Tiri Salvezza}
\index{TS|see {Tiri Salvezza}}
Perché un TS riesca, devi tirare d20 e ottenere un risultato uguale o inferiore al Punteggio di Abilità appropriato. 1 è sempre un successo e 20 è \mbox{sempre} un fallimento.

\vfill

\paragraph{Vantaggio e Svantaggio}
\index{Vantaggio}
\index{Svantaggio}
Tutte le volte che le probabilità di successo in un TS sono accresciute o diminuite, l’Arbitro potrebbe attribuire Vantaggio o Svantaggio: tira due volte e prendi rispettivamente il risultato migliore o quello peggiore. Vantaggio e Svantaggio si cancellano l’un l’altro. 

\vfill

\paragraph{Svolgere il proprio Turno}
\index{Combattimento}
\index{Turni}
In una situazione di combattimento, l’Arbitro decide quale parte agisce per prima. Quando questo non è~chiaro, i personaggi non giocati dall'Arbitro devono superare un \saves{DES} per poter agire prima della \mbox{loro} controparte. Poi, terminato questo turno iniziale, agiscono insieme come di consueto.

Al proprio turno, i personaggi possono di norma \textbf{muoversi} dall’attuale posizione (o in alternativa sostituire gli oggetti che impugnano) e \textbf{poi} eseguire una singola \textbf{azione}. Tutti i personaggi dichiarano le rispettive \mbox{intenzioni} e \textbf{solo dopo} vengono tirati i dadi. 

\vfill

\paragraph{Attaccare}
\index{Attacchi}
\index{Danno}
\index{Danno!bonus}
\index{Danno Bonus|see {Danno, bonus}}
Tira il dado del Danno della tua arma, o quelli di entrambe le armi se ne brandisci due, unitamente a qualsivoglia dado bonus in tuo possesso. Letto il singolo tiro più alto, l’attacco infligge quel quantitativo di Danno.

Le armi a distanza non possono essere usate mentre si è impegnati in un combattimento in mischia.

\vfill

\paragraph{Coalizzarsi contro un Solo Bersaglio }
\index{Coalizzarsi contro un Solo Bersaglio }
\index{Bypassare i PF}
 Quando più attaccanti prendono come bersaglio un singolo individuo, ogni attaccante tira per sé: va quindi tenuto il risultato dell’attaccante che ha fatto il tiro più alto, a cui va aggiunto 1 punto di Danno per ogni ulteriore attaccante, fino a un massimo di +5. Risolto l’attacco, quel bersaglio non può essere attaccato di nuovo fino al suo prossimo turno. 

Quando qualcuno di questi attacchi bersaglia direttamente i Punteggi di Abilità, gli attacchi vanno raggruppati per Punteggio di Abilità attaccato e risolti separatamente dagli attacchi ordinari secondo la regola del Coalizzarsi contro un Singolo Bersaglio.

\vfill

\begin{dbox}
\paragraph{Coalizzarsi contro un Solo Bersaglio: Modalità Facile (opzionale)}
Per un combattimento più “cinematografico”, potreste rinunciare al Danno bonus previsto per ogni attaccante in più.
\end{dbox}

\vfill
\break

\paragraph{Attacchi Compromessi e Potenziati}
\index{Attacchi!compromessi}
\index{Attacchi Compromessi|see {Attacchi, compromessi}}
Gli attacchi che sono Compromessi, come il fare fuoco attraverso una copertura o l’attaccare un bersaglio resistente, tirano d4 Danni a prescindere dall’arma e non sono concessi dadi bonus per il Danno.

\index{Attacchi!potenziati}
\index{Attacchi Potenziati|see {Attacchi, potenziati}}
Gli attacchi che sono Potenziati da un rischioso espediente o in virtù di un bersaglio vulnerabile ottengono un d12 come dado bonus per il Danno.

Potenziamento e Compromissione si annullano a vicenda.

\vfill
\paragraph{Manovre}
\index{Manovre}
Piuttosto che eseguire il tuo normale attacco, potresti impiegare il tuo turno per tentare di eseguire un’altra manovra, come buttare a terra un bersaglio avversario, sgraffignare un oggetto o fuggire. In casi simili, la parte maggiormente a rischio esegue un TS per evitare le conseguenze.

\vfill
\paragraph{Armatura}
\index{Armatura}
Un’armatura sottrae il suo punteggio dal risultato di qualsiasi tiro per il Danno fatto contro chi la sta indossando. 

\index{Bypassare i PF}
Se il danno bypassa i PF è comunque soggetto all’Armatura del bersaglio a meno che non sia indicato diversamente. 

Il punteggio totale di Armatura di una creatura non può essere maggiore di 3.

\vfill
\paragraph{Combattere in Sella}
\index{Combattimento!in sella}
\index{Combattimento in Sella|see {Combattimento, in sella}}

Truppe in mischia in groppa a cavalcature ottengono +1~Armatura e un dado bonus per il Danno contro opponenti a piedi, ma di norma non possono impiegare armi a due mani.

\vfill
\paragraph{Danno}
\index{Danno}
\index{Punti Ferita}
Quando prendi del Danno, i Punti Ferita vanno ridotti di un pari ammontare. Se i PF non si sono azzerati, allora l’attacco è stato perlopiù evitato oppure è stata inflitta solo una ferita marginale.

Quando i tuoi PF si azzerano, ogni Danno rimasto va a ridurre il tuo punteggio di FOR. Ora, devi superare un \save{FOR} per evitare il Danno Critico.

\vfill
\paragraph{Danno da Scoppio}
\index{Danno!da scoppio}
\index{Danno da Scoppio|see {Danno, da scoppio}}
Gli attacchi da Scoppio colpiscono tutti i bersagli \mbox{nell’area} appropriata, tirando una volta per ciascuno di essi. Se si hanno dubbi su quanti bersagli possano essere coinvolti, stabilitelo tirando il dado del Danno.

\vfill
\paragraph{Danno Critico}
\index{Danno!critico}
\index{Danno Critico|see {Danno, critico}}
I personaggi che subiscono Danno Critico sono impossibilitati a compiere ulteriori azioni finché un soggetto alleato non decida di prendersene cura e non facciano un Riposo. Se lasciati senza assistenza per un’ora, \textbf{muoiono}.

\vfill
\paragraph{Perdita di Punti Abilità}
\index{Perdita di Punti Abilità}
Un personaggio \textbf{muore} a FOR~0. A DES~0 o a VOL~0 è~\mbox{rispettivamente} \textbf{paralizzato} o \textbf{catatonico}, non può \mbox{agire} prima di una Guarigione e deve essere condotto in un luogo sicuro.

\vfill
\paragraph{Morte}
\index{Morte}
Se un personaggio muore, chi lo giocava ne crea uno nuovo e l’Arbitro escogita un modo per farlo unire alla compagnia il prima possibile. Un'alternativa potrebbe essere prendere il controllo di un personaggio Aiutante o Apprendista.

\vfill
\paragraph{Morale}
\index{Morale}
Per evitare che il suo gruppo venga messo in rotta quando perde la metà dei suoi membri, il capo del gruppo deve superare un \save{VOL}. Gli individui che sono da soli devono superare questo Tiro Salvezza quando arrivano a 0pf. Quanto detto si applica a creature nemiche o alleate, ma non ai personaggi-giocanti e nemmeno a~opponenti incapaci di ragionare o di provare paura.

\vfill
\paragraph{Ritirarsi}
\index{Ritirarsi}
Fuggire per mettersi in salvo durante un inseguimento richiede un \save{DES} e un luogo verso cui correre.

\vfill
\paragraph{Riposo}
\index{Riposo}
Qualche minuto di Riposo e un sorso d’acqua fanno recuperare tutti Punti Ferita persi, ma potrebbe far sprecare tempo prezioso o attirare pericoli.

Inoltre, il Riposo potrebbe essere proibito per clima avverso, ambiente ostile, mancanza di razioni, ecc.

\vfill
\paragraph{Guarigione}
\index{Guarigione}
Per recuperare Punti Abilità Persi e riprendersi da altre problematiche serie di salute occorre un'assistenza professionale o il ricorso alla magia.

\vfill
\paragraph{Punteggi di Abilità Impliciti}
\index{Punteggi di Abilità}
Un Punteggio di Abilità non indicato va inteso pari a 10.

\vfill
\paragraph{Reazione}
\index{Reazione}
Se la reazione verso un personaggio è \textbf{incerta}, per evitarne una negativa si deve superare un \save{VOL}.

\vfill
\paragraph{Compagni Animali}
Solo uno per personaggio e segue semplici comandi.

\vfill
\paragraphsection{Condizioni:}
\index{Condizioni}

\index{Accecato|see {Condizioni}}
\subparagraph{Accecato:} il soggetto potrebbe dover superare un \save{DES} per riuscire a completare azioni basate sulla vista e i suoi attacchi sono Compromessi; i \saves{DES} dovuti a minacce esterne sono tirati con Svantaggio.

\vfill
\index{Frenato|see {Condizioni}}
\subparagraph{Frenato:} il soggetto ha Svantaggio ai \saves{DES} e gli attacchi contro di esso sono Potenziati.

\vfill
\index{Invisibile|see {Condizioni}}
\subparagraph{Invisibile:} gli attacchi del soggetto sono Potenziati e~quelli contro le creature Invisibili sono Compromessi.

\vfill
\index{Nascosto|see {Condizioni}}
\subparagraph{Nascosto:} gli attacchi del soggetto sono Potenziati, ma qualsiasi attacco o altra azione simile rivela l’attaccante.

\vfill
\index{Stordito|see {Condizioni}}
\subparagraph{Stordito:} il soggetto è Frenato e non può agire.

\vfill
\index{Svenuto|see {Condizioni}}
\subparagraph{Svenuto:} i PF del soggetto si riducono a 0.

\break

\section{Dopo l'Avventura}
\index{Avventura}
\index{Avanzamento}

Di solito, l’obiettivo di un personaggio Avventuriero è~scoprire un luogo avvolto nel mistero, annientare una potente minaccia o andare alla ricerca di oscuri tesori.

\vfill
\subsection{Livelli d’Esperienza}
\index{Livelli d’Esperienza}
\index{Livelli|see {Livelli d’Esperienza}}
Dopo aver soddisfatto i requisiti per il prossimo Livello d’Esperienza, potresti prenderti una pausa dall’Avventura per riflettere sulla tua esperienza. Descrivi che cosa ha fatto il personaggio durante questo periodo, a~prescindere se sia stato qualcosa di umile o di grandioso. Dopodiché, passi al Livello d’Esperienza successivo. Non puoi passare più di un singolo Livello d’Esperienza per sessione di gioco.

\index{Tratti}
Quando passi a un nuovo Livello d’Esperienza scegli un nuovo \textbf{Tratto}, ottieni \textbf{d6pf} e tiri \textbf{d20 per ciascun Punteggio di Abilità}: se il tiro è più alto del Punteggio di Abilità, quest’ultimo aumenta di un punto fino a un massimo di 20. 

Se nessun Punteggio di Abilità è cambiato, aumenta il tuo Punteggio di Abilità più basso di un punto fino al massimo di 20.

\paragraph{1. Novizio}
\index{Novizio|see {Livelli d’Esperienza}}
Il tuo personaggio è pronto per la sua prima Avventura.

\paragraph{2. Riconosciuto}
\index{Riconosciuto|see {Livelli d’Esperienza}}
Il tuo personaggio è sopravvissuto ad almeno \textbf{una} Avventura verso un luogo pericoloso prima di far ritorno alla civiltà. 

\paragraph{3. Esperto}
\index{Esperto|see {Livelli d’Esperienza}}
\index{Apprendista}
Il tuo personaggio è sopravvissuto ad almeno \textbf{tre} Avventure da quando ha raggiunto il Livello Riconosciuto. 

Puoi adesso assegnargli un personaggio \textbf{Apprendista} creato allo stesso modo di un nuovo personaggio-giocante. 

\paragraph{4. Veterano}
\index{Veterano|see {Livelli d’Esperienza}}
Il tuo personaggio è sopravvissuto ad almeno \textbf{cinque} Avventure da quando ha raggiunto il Livello Esperto. Adesso, ha un personaggio \textbf{Apprendista} che ha raggiunto il Livello Esperto. 

\paragraph{5. Maestro}
\index{Maestro|see {Livelli d’Esperienza}}
Il tuo personaggio ha fondato un Possedimento di almeno cento persone o se n’è impadronito. Gli viene concesso un titolo nobiliare o ne crea uno suo. Gli altri personaggi potrebbero assisterlo nel raggiungimento di questo obiettivo, sebbene in questi casi ciò renderebbe Maestro soltanto il tuo personaggio. 

\vfill
\begin{dbox}
	\paragraph{Progressioni Alternative per l’Esperienza (opzionale)}
	Se la progressione di 1--3--5 Avventure sembra troppo rapida, usate al suo posto la progressione di 3--5--7. 

	Se giocate un grosso modulo non ripartibile, il passaggio di Livello costerà oro e tesori guadagnati in Avventura e da spendere in addestramento: 1f--5f--25f--125f. 
\end{dbox}

\end{document}
