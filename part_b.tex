\documentclass[itdr]{subfiles}

\begin{document}

\chapterx{Appendice B: Bestiario}
\label{ch:appendice_b}
\index{Mostri}

L'Arbitro dovrebbe usare questi esempi come guida e fonte d'ispirazione per la creazione dei suoi mostri.

\vfill

\statpar{Contemplatore}
FOR~16, DES~16, VOL~17, 20pf, Armatura~1.

Cerca attivamente di distruggere ogni altra forma di vita. Sotto lo sguardo del Contemplatore la magia non funziona. Ogni turno, è in grado di sparare a bersagli diversi due tra i raggi qui in basso.

\subparagraph{Raggio Telecinetico:} un bersaglio non più grande di un elefante viene sollevato, spostato o scagliato via. I bersagli viventi lanciati in questo modo prendono d6 Danno, ma gli oggetti scagliati potrebbero causarne fino a d12, a seconda delle dimensioni.

\subparagraph{Raggio Terrorizzante:} \save{VOL} o si cade in preda al terrore. Se al suo turno successivo il bersaglio non resta congelato sul posto o non scappa, Perde d6~VOL.

\subparagraph{Raggio Disintegrante:} d6 Danno che ignora l'\mbox{Armatura}. Chiunque prenda Danno Critico si riduce in polvere. Distrugge completamente gli oggetti statici fino alle dimensioni di un elefante.

\vfill

\statpar{Cubo Gelatinoso}
FOR~14, DES~3, VOL~3, 16pf, Armatura~2.

Finché non lo si osserva pericolosamente da vicino, il Cubo pare un ammasso di umida aria fumosa. Un odore di prodotti chimici potrebbe tradire la sua natura da una distanza maggiore. Il Cubo è attratto da rumore e calore.

Non compie attacchi normali: qualunque individuo venga investito dal Cubo ne è inglobato a meno che non superi un \save{DES} per scansarlo con un salto, ammesso che ci sia lo spazio per farlo. I soggetti inglobati Perdono d8~DES ogni turno e d6~FOR ogni ora, finché non sono digeriti. Non possono liberarsi da soli, ma devo essere tirati fuori dal Cubo con altri mezzi. Quando il Cubo prende Danno Critico, collassa in una pozzanghera di poltiglia appiccicosa.

\vfill

\statpar{Drago Rosso}
FOR~20, VOL~12, 25pf, Armatura~3, 2d10~Artigli, \mbox{Vantaggio} ai Tiri Salvezza contro magia, volo.

Astuti e pericolosi rettili dalle enormi dimensioni. Sono in grado di parlare, ma in genere evitano di farlo a meno che non vengano debitamente motivati.

I Draghi Rossi hanno l'istinto di accumulare tesori, in particolare gli oggetti in oro. Il tesoro di un drago varrà 5d20f. Se raccolte come si deve, le parti di un drago morto potranno essere vendute ad acquirenti ad hoc.

\subparagraph{Soffio di Fuoco:} d6 Danno da Fuoco a chiunque si trovi all'interno della deflagrazione. Infligge inoltre d6 Danni da Fuoco alla fine del suo prossimo turno a meno che non si superato un \save{DES} o non si sia trovato un modo di spegnere le fiamme.

\vfill

\statpar{Ghoul}
FOR 13, DES 15, VOL 6, 5pf, 2d6 Artigli, d8 Morso, \mbox{immunità} agli Incantesimi che alterano la mente.

Esseri mostruosi che abitano i cimiteri abbandonati e che si nutrono di carne umana, viva o morta.

\subparagraph{Tocco del Ghoul:} se gli Artigli riducono il Punteggio di FOR, \mbox{il bersaglio} è Stordito finché non supera un \save{FOR} alla fine del suo turno; nel mentre emana un fetore che disgusta chi si trova nelle sue vicinanze.

\vfill

\statpar{Gnoll}
FOR~12, DES~12, VOL~7, 9pf, Armatura~2 \mbox{(leggera + scudo),} ascia (d6/d8), refurtiva di 3d6s in monete e ninnoli.

Il loro fetore è molto riconoscibile e si diffonde nell'ambiente circostante.
1 su 6 possibilità di non essere ostili. Ci si può negoziare, ma sono davvero interessati solo alla carne, preferibilmente viva.

\vfill

\statpar{Goblin}
FOR~8, DES~14, VOL~8, 4pf, lancia (d6), arco (d4).

Creature maliziose facilmente corruttibili con oggetti che considerano belli. Alcuni goblin possono lanciare Trucchetti.

\vfill

\statpar{Imp}
FOR~6, DES~16, VOL~14, 3pf, d6~Morso Velenoso,\\tutti gli attacchi (eccetto quelli con armi magiche) sono Compromessi.

Un diavoletto alato ingannatore. Può usare la sua azione per lanciare a volontà Individuazione del Magico e Invisibilità e Suggestione una volta a Riposo.

\subparagraph{Morso Velenoso:} se il morso riduce il punteggio di FOR, \mbox{il bersaglio} subisce anche la Perdita di 1 DES.

\subparagraph{Mutazione della Forma:} cambia il suo aspetto esteriore in quello di un piccolo animale.

\vfill

\dimage{filtheater2}{120pt}

\vspace{-2ex}

\statpar{Ingozza Lerciume}
FOR~16, DES~6, VOL~5, 16pf, Armatura~1, d6~Morso.

Grosse e stupide bestie che mangiano pressoché qualsiasi cosa capiti loro a tiro. Apprezzano di più la carne morta rispetto a quella viva. Tra un latrato e l'altro, possono sfoggiare uno scarso vocabolario della lingua comune, ma hanno bassa capacità di comprensione.

\subparagraph{Danno Critico:} se non supera un \save{FOR}, il bersaglio contrae la {\em febbre da lerciume}. Se lo fallisce, il giorno dopo sarà molto malato e non avrà benefici dai Riposi.

\vfill

\statpar{Landasqualo}
FOR~17, DES~8, VOL~8, 18pf, Armatura~3, d8~Morso.

Incide la terra come fosse acqua, sfuttando ciò per tendere agguati alle sue prede. Se teme per la sua vita, un Landasqualo potrebbe causare una frana. La caduta dei massi fa d6 Danno, ma chiunque resterà così a lungo da farsi seppellire prenderà d10 Danno. Prima di allora, il mostro si sarà già allontanto scavando un cunicolo.

\vfill

\statpar{Manticora}
FOR~17, DES~15, 8pf, Armatura~1, 2d6~Artigli, d8~Morso.

Un orribile abominio con il corpo di un leone, una testa umana zannuta e una coda piena di aculei.

\subparagraph{Aculei della Coda:} questi aculei velenosi possono essere sparati abbastanza lontano, infliggendo d6~Danno. Se l'aculeo riduce il punteggio di FOR, il bersaglio subisce anche la Perdita di d4 DES.

\vfill

\statpar{Mastino Infernale}
DES~12, 5pf, Armatura~1, d6~Morso, immunità al Fuoco.

Neri e diabolici cani fiammeggianti che cacciano in branchi.

\subparagraph{Soffio di Fuoco:} d4 Danno da Fuoco in un cono piccolo.

\vfill

\statpar{Mummia}
FOR~16, DES~8, 9pf, Armatura~1, d8~Pugno, immune agli attacchi non magici, gli attacchi con il Fuoco sono Potenziati.

Vendicativi cadaveri imbalsamati risvegliati da incauti tombaroli tra le rovine di antiche città o templi.

I soggetti sorpresi da una Mummia devono superare un \save{VOL} o sono Storditi per il prossimo turno.

\subparagraph{Danno Critico:} infetta il bersaglio con la {\em putrefazione della mummia}. Il punteggio di FOR e il massimale di pf del bersaglio infetto sono ridotti immediatamente d4~punti e poi di nuovo a ogni giorno successivo, finché il bersaglio non è curato grazie alla rimozione della maledizione o a una settimana di cure convenzionali.

\break

\begin{center}
\includegraphics[width=0.85\linewidth]{img/pic/snakedemon.png}
\end{center}
%\dimage{snakedemon}{250pt}

\vspace{-2ex}

\statpar{Ofidio-Demone}
FOR~17, DES~17, VOL~16, 15pf, Armatura~1, sei spade (6d6, può bersagliare molteplici opponenti in mischia).

Supervisionano le attività operative dei demoni e guidano gli sgherri minori. Amano combattere in singolar tenzone e mai rinunciano a un duello. Possono lanciare come azione gli Incantesimi qui in basso.

\subparagraph{Volo Veleggiato:} chi lo lancia può volare abbastanza rapidamente finché non tocca terra o prende Danno.

\subparagraph{Barriera di Anime:} una barriera fatta di visioni di anime straziate urlanti e inveenti. Chi l'attraversa prende 8 Danno e Perde d6~VOL in caso di Danno Critico.

\vfill

\statpar{Orchetto}
FOR~11, DES~9, VOL~8, 5pf, Armatura~2 (Armatura frammentaria + scudo), arma da guerra (d6/d8).

Immorali tirapiedi visti di rado non al servizio di qualche ignobile leader e che variano enormemente nell'aspetto esteriore a seconda di chi sia il loro capo.

\vfill

\statpar{Orco}
FOR~18, DES~8, VOL~7, 6pf, Armatura~1, clava (d8).

Grossi e scorbutici bruti mangia-uomini.

\vfill

\statpar{Orrore Uncinato}
FOR~15, DES~8, VOL~6, 7pf, Armatura~3, 2d8~Uncini.

Questo Orrore alto dieci piedi si aggira per gallerie e grotte, usando il suo ticchettio sonoro come forma di ecolocalizzazione. La sua vista è molto scarsa e viene disorientato facilmente dai rumori forti.

Qualunque cosa delle dimensioni di un cane o più piccola rappresenta potenziale cibo da ingoiare per intero quando infligge Danno Critico, che causa d6~Punti di FOR Persi dopo la deglutizione. L'essere considererà qualsisi creatura più grande di così come una minaccia per il suo territorio e si batterà ferocemente, ma eviterà qualsiasi cosa più grossa di lui.

\break

\statpar{Orso Bubbolante}
FOR~15, DES~6, VOL~5, 10pf, Armatura~1, 2d8~Artigli.

Emette constantemente un bubbolio subsonico, usato per percepire l'ambiente circostante: per questo, non puoi mai coglierlo di sorpresa, a meno che il suo udito non sia in qualche modo indebolito.
Può emettere un singolo bubbolio che scuote le ossa e fa d6 Danno a chi è lì vicino: un bersaglio ridotto a 0pf dal bubbolio non è a rischio di Danno Critico, ma deve superare un \save{FOR} o è Stordito per il prossimo turno.

\vfill

\statpar{Pantera Fase}
FOR~16, DES~18, VOL~6, 13pf, 2d8~Tentacoli~Unghiati.

La sua immagine sfasata dà a questa bestia Vantaggio ai Tiri Salvezza da Danno Critico. Attacca senza motivo ogni altro essere vivente, per puro divertimento.

\vfill

\statpar{Rana Fetida}
DES~13, VOL~7, 6pf, Armatura~1, lancia (d8).

Attacca senza motivo e di solito tende imboscate ai suoi bersagli. È anfibia e può coprire con un salto una distanza pari a diverse volte la sua altezza. Gli animali comumi mostrano una forte animosità verso le Rane Fetide e le attaccheranno con lo scopo di allontanarle.

\vfill

\statpar{Rugginofago}
DES~12, VOL~5, 6pf, d6~Morso.

Di solito non attacca. È capace di trasformare il metallo in una polvere simile a ruggine, di cui poi si nutre. Se chi gli si oppone in mischia ha con sé un'arma di metallo, uno scudo o un'armatura, il Rugginofago trasformerà uno di questi oggetti in ruggine con un'azione, a meno che non si superi un \save{DES}.

\vfill

\statpar{Scheletro}
DES~13, VOL~12, 5pf, Armatura~2 (solo contro attacchi perforanti come frecce e lance), spada spuntata (d6).

Quando dovrebbe essere ucciso da attacchi fisici, si frantuma in almeno due parti distinte che, se non tenute separate, riformano lo scheletro al suo prossimo turno, lasciandolo a 0pf. Ciascuna metà continua a combattere, ma quella senza spada infligge solo d4 Danno.

\vfill

\statpar{Signore dei Cervelli}
FOR~14, DES~14, VOL~20, 18pf, immunità agli Incantesimi che alterano la mente.

Grazie alla sua capacità psichica può levitare, proiettarrsi verso altre realtà e impartire con la telepatia qualunque comando: il bersaglio che si rifiuta di obbedirgli, deve superare un \save{VOL} o Perde d8~VOL.

\subparagraph{Deflagrazione Mentale:} attacca la mente del bersaglio con energia psichica per d8 Danno. Il Danno Critico da questo attacco influenza la VOL invece della FOR ed è evitato con un \save{VOL}.

\subparagraph{Danno Critico in Mischia:} il cervello del bersaglio viene estratto e mangiato. Il Signore dei Cervelli assorbe i suoi ricordi più recenti.

\break

\statpar{Troll}
FOR~18, DES~13, VOL~7, 9pf, 3d8~Artigli e Morso (può bersagliare molteplici opponenti in mischia).

Gigantesche creature umanoidi senza paura e con una passione per la carne.

\subparagraph{Mutazioni:} in virtù della loro innaturale rigenerazione, ad alcuni Troll crescono arti extra, altre teste oppure sviluppano deformità addirittura più assurde.

\subparagraph{Rigenerazione:} recupera d6pf, d6~Punti~di For Persi e si riprende dal Danno Critico all'inizio di ogni turno. La rigenerazione non funziona se il Troll ha preso Danno da Acido o da Fuoco nel turno precedente.

\vfill

\dimage{purpleworm}{234pt}

\vspace{-1ex}

\statpar{Verme Purpureo}
FOR~20, DES~3, VOL~5, 30pf, Armatura~3, d10~Pungiglione.

\subparagraph{Danno Critico:} il bersaglio è punto e Perde 3d6~FOR.

Si muove sottoterra lasciando sulla sua scia tunnel circolari. Può provare a ingoiare una creatura di taglia media o più piccola: il bersaglio deve superare un \save{DES} o è ingoiato per intero, Perdendo d10~DES ogni turno e d8~FOR ogni ora mentre è digerito. Quando tira per Danno Critico, il Verme deve superare un \save{FOR} in più o rigurgita ogni creatura ingoiata.

\vfill

\statpar{Yeti}
FOR~18, DES~14, 6pf, Armatura~1, 2d6~Artigli.

Abominevoli gorilla giganti che di solito vivono in foreste d'alta montagna e cacciano tendendo imboscate.

\subparagraph{Presa:} il bersaglio deve superare un \save{DES} o è Frenato e prende d8~Danno ora e a ogni turno successivo finché non passa un \save{FOR o DES}.

\subparagraph{Sguardo Terrificante:} quando lo Yeti si mostra e osserva chi gli si oppone, bisogna superare un \save{VOL} o si prende la condizione Stordito per il prossimo turno.

\break

\statpar{Zombi}
FOR~14, DES~6, 3pf, d6~Pugno, lento, una volta per Riposo \mbox{ignora} l'effetto del primo Danno Critico che subisce.

Un cadavere che cammina animato dalla magia.

\section{Animali Normali}

\statpar{Alce}
FOR~16, VOL~5, 6pf, d8~Corna.

\statpar{Cervo}
DES~16, VOL~5, 2pf, d6~Zoccoli.

\statpar{Cinghiale Selvatico}
FOR~13, DES~11, VOL~5, 4pf, d6~Zanne.
\subparagraph{Carica:} o si supera un \save{DES} o si prende d8~Danno.

\statpar{Coccodrillo}
FOR~15, VOL~5, 3pf, Armatura~1, d8~Morso.

\statpar{Elefante}
FOR~20, VOL~8, 12pf, Armatura~1, d10~Zanne.
\subparagraph{Carica:} il bersaglio deve superare un \save{DES} o prende il Danno delle Zanne ed è sbattuto a terra prono.
\subparagraph{Calpestamento:} un bersaglio prono prende d12~Danno.

\statpar{Leone}
FOR~17, DES~15, 6pf, Armatura~1, 2d6~Artigli, d8~Morso.
\subparagraph{Balzo:} o si supera un \save{DES} o si prende posizione prona subendo gli attacchi combinati di Artigli e Morso.

\statpar{Lupo}
FOR~12, DES~15, VOL~6, 3pf, d6~Morso.

\statpar{Orso}
FOR~15, VOL~7, 6pf, Armatura~1, 2d6~Artigli, d8~Morso.

\statpar{Serpente Costrittore}
FOR 16, VOL~3, 5pf, d4~Morso.
\subparagraph{Stritolamento:} il bersaglio deve superare un \save{DES} o è Frenato e prende d8~Danno ora e a ogni turno successivo finché non passa un \save{FOR o DES}.

\statpar{Serpente Velenoso}
DES~16, VOL~3, 3pf, d6~Morso Velenoso.
\subparagraph{Morso Velenoso:} se il morso riduce il punteggio di FOR, il bersaglio subisce anche la Perdita di d4 DES.
~
\end{document}
