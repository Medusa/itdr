\documentclass[itdr]{subfiles}

\begin{document}

\chapter{Possedimenti}
\label{Possedimenti}
\index{Possedimenti}

Qualsiasi comunità di 100 o più persone è un Possedimento. Uno o più personaggi potrebbero avere il controllo di un Possedimento, con il potenziale d’appropriarsi stabilmente di una parte di mondo.

%\vfill

\section{Punteggi di Taglia e Popolazione}
\index{Punteggio di Taglia}
\index{TAG|see {Punteggio di Taglia}}
\index{Popolazione}

Il Punteggio di Taglia, o TAG, è una misura della popolazione del tuo Possedimento e potrebbe arrivare fina a~un massimo di 20 punti.

\begin{dtable}[Lr|Lr|Lr]
	\textbf{TAG} & \textbf{Abitanti} & \textbf{TAG} & \textbf{Abitanti} & \textbf{TAG} & \textbf{Abitanti}\\
	0 & <100 & 7	& 7.500		& 14 & 100.000 \\
	1 & 100	& 8	& 10.000	& 15 & 150.000 \\
	2 & 300	& 9	& 15.000	& 16 & 200.000 \\
	3 & 600	& 10& 20.000	& 17 & 300.000 \\
	4 & 1.000	& 11& 30.000	& 18 & 500.000 \\
	5 & 3.000	& 12& 50.000	& 19 & 750.000 \\
	6 & 5.000	& 13& 75.000	& 20 & 1.000.000 \\
\end{dtable}

\index{Interesse del Possedimento}
All'\textbf{inizio} di ogni mese, scegli un Interesse del Possedimento. Questo obiettivo è raggiunto alla \textbf{fine} del mese:

\begin{itemize}
	\item \textbf{Tassazione:} questo mese raccogli denaro extra, guadagnando 1s per ogni abitante del Possedimento.
	\item \textbf{Crescita:} tira d20. Se il risultato è più alto del tuo TAG, alla il tuo TAG aumenta di 1.
	\item \textbf{Reclutamento:} recluti un’armata (vedi \textbf{Addestramento Milizia}, in basso). Non puoi ripetere Reclutamento finché il tuo TAG non aumenta. Il tuo prossimo tiro per Crescita avrà Svantaggio.
	\item \textbf{Prosperità:} questo mese non hai bisogno di tirare per controllare se c’è Malcontento nel tuo Possedimento.
\end{itemize}

\index{Malcontento}
\subparagraph{Malcontento:} Alla \textbf{fine} del mese, tira d20. Se il risultato è più basso del tuo TAG, c’è Malcontento nel tuo Possedimento. Il 10\% della popolazione si rivolta e deve essere domato, altrimenti s’impadronirà del controllo del tuo Possedimento. 

\vfill
\section{Armate e Guerra}
\index{Combattimento!combattimento di massa}
\index{Armate|see {Combattimento, combattimento di massa}}
\index{Battaglie|see {Combattimento, combattimento di massa}}
\index{Grandi Battaglie|see {Combattimento, combattimento di massa}}
\index{Combattimento di Massa|see {Combattimento, combattimento di massa}}
\index{Guerra|see {Combattimento, combattimento di massa}}

\index{Unità|see {Combattimento, combattimento di massa}}
\index{Gruppi Grandi}

\index{Milizia}
\subparagraph{Addestramento Milizia:} gli individui che compongono il 20\% della popolazione sono idonei a essere chiamato in servizio come reclute poco qualificate (3pf). Un ulteriore 1\% della tua popolazione è composto da militi di professione (FOR 12, 5pf, Combattente Novizio). Tutte le truppe devono essere equipaggiate come richiesto.

Un’armata che vince una battaglia contro una pari o più forte armata avversaria può essere ulteriormente addestrata al ritmo di 1\% della tua popolazione al mese. 

Le reclute diventano militi (1s/persona) e i \mbox{militi} diventano difensori (10s/persona) (FOR~14, 10pf, Combattente Riconosciuto).

\vfill
\break

\subparagraph{Grandi Battaglie:} quando avete a che fare con grandi numeri di individui impegnati in una battaglia (in genere, 10 o più), essi dovrebbero essere raggruppati assieme a formare un’unità. Le unità hanno gli stessi Punti Ferita di un singolo individuo di quell’unità, ma aggiungono 1 Danno per ogni singola volta che doppiano l’opponente (o sottraggono, se sono in inferiorità numerica), da -5 a +5. P. es., un’unità di 200 cavernicoli che combatte contro 50 lancieri supera in numero quest’ultima \mbox{4 a 1}, ottenendo così 4 Danni bonus.

Quando le unità subiscono Danno Critico, i loro numeri vanno dimezzati e devono superare un \save{VOL} o si rompono per poi sciogliersi. A FOR~0 sono spazzate via.

Gli attacchi individuali contro le unità sono Compromessi.

Gli attacchi delle unità conto individui singoli sono Potenziati, aggiungono +5 di bonus al Danno e~infliggono Danno a Scoppio.

Gli attacchi delle unità che fanno Danno da Scoppio contro altre unità, hanno un dado bonus del Danno dell’arma.

\index{Assedi}
\index{Mura}
\index{Strutture}
\subparagraph{Assedi:} le mura di legno hanno 6pf e Armatura~6 mentre le mura di pietra hanno 8pf e Armatura~8. Ridurre delle mura a 0pf consente di poterle superare. Le mura e le altre strutture difensive in genere ignorano qualsiasi Danno che non provenga da macchine d’assedio e simili.

\index{Macchine d'Assedio}
\subparagraph{Macchine d’Assedio:} cannoni e simili infliggono d12 Danno da Scoppio.

\begin{dbox}
	Vedi \textbf{\customref{sec:strutture_e_assedi}{Strutture e Assedi}} nell'\textbf{\customref{ch:appendice_a}{Appendice A}} per maggiori dettagli e ulteriori informazioni.
\end{dbox}

\vfill

\section{Esempi di Possedimenti}

\paragraph{Collina Rossa --- Casa delle Bestie-Uomo}
Governante: Yur Og il Nero, Sciamano Veterano.\\
TAG 5 (Abitanti 3.000).\\
Mura di pietra (8pf, Armatura~8), 4 Lanciatori di Pietre. 30 Difensori della Tribù (ascia a due mani), 300 Uomini Selvaggi (ascia, scudo), 300 Uomini Selvaggi (arco).

\paragraph{Unktar --- La Città d’Argilla delle Mosche}
Governante: Primarca Elm Vroach, Sacerdote Maestro.\\
TAG 14 (Abitanti 100.000).\\
Mura d’argilla (7pf, Armatura~7), 10 Versatori d’Olio Bollente, 10 Cannoni. 5.000 Lancieri (lancia, scudo), 6.000 arcieri (arco), 2.000 alabardieri (alabarda, armatura leggera), 2.000 Cavalieri Leggeri (cavallo, lancia, arco), 2.000 Arcieri Nomadi (armatura leggera, arco lungo), 800 Guardie della Sala Grande (cavallo, armatura completa, spadone).

\vfill

\end{document}
