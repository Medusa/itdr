\documentclass[itdr]{subfiles}

\begin{document}

\chapter{Esempio di Giocata}
\label{ch:esempio_di_giocata}
{\setstretch{1.2}

{\em Tre personaggi-giocanti e il loro aiutante tedoforo si sono addentrati nelle profondità di un singolare complesso sotterraneo in cui si sono imbattuti all'interno di un inospitale deserto.}

\vspace{1ex}

\subparagraph{Arbitro:} la base della lunga scalinata conduce in una sala spettacolare, alta circa 30~piedi e larga altrettanto. Le sue pareti sembrano un mosaico intricato, ma le tessere cambiano di continuo di colore. Ondate di varie tonalità si infrangono sulle pareti e il centro del pavimento è dominato da un pozzo circolare largo sei piedi.

\subparagraph{Ezekiel (Mistico):} {\em (abbozzando la stanza sulla sua mappa approssimativa)} ci sono altre uscite oltre la strada da cui veniamo noi?

\subparagraph{Arbitro:} solo il pozzo al centro della stanza.

\subparagraph{Toku (Combattente):} be', è un vicolo cieco. Il mio istinto di cacciatore ci aveva visto giusto!

\subparagraph{Ezekiel:} le pareti hanno un aspetto assurdo\ldots Sto molto attento a non toccarle e dico anche al mio tedoforo di fare altrettanto.

\subparagraph{Toku:} dai, lo abbiamo assoldato perché è sacrificabile! Forse Uthred potrebbe provare a toccarle.

\subparagraph{Uthred (Combattente):} non ho paura delle pareti, ma non sono stupido. Proverò a picchiettare sui muri con il manico della mia ascia.

\subparagraph{Arbitro:} la trama delle tessere non sembra reagire, ma visto che la stai esaminando più da vicino, riesci a percepire che le tessere emanano un leggero calore.

\subparagraph{Uthred:} abbastanza da potermi scottare?

\subparagraph{Arbitro:} non sembra, sono soltanto un po' calde.

\subparagraph{Uthred:} piazzo con coraggio la mano sulle tessere.

\subparagraph{Arbitro:} non appena la mano di Uthred tocca la parete, i colori mutevoli si fermano e una trama blu pulsante inizia a irradiarsi dalla mano di Uthred.

\subparagraph{Ezekiel:} ohi ohi! Ora gli esplode la testa\ldots

\subparagraph{Uthred:} ti stai preoccupando troppo! Che sento toccando le piastrelle?

\subparagraph{Arbitro:} la sensazione è quella che ti aspetteresti toccando un mosaico liscio, ma emanano un lieve tepore.

\subparagraph{Uthred:} uh, strano. Be', io levo la mano dalla parete e me ne vado a controllare il pozzo.

\vfill
\break

\subparagraph{Arbitro:} quando togli la mano dal muro, i colori riprendono a mutare e adesso vedi la figura piastrellata di una persona, che pare quasi il tuo riflesso. Dopo appena un secondo, la stanza si riempie di crepitii e le sembianze piastrellate di Uthred escono in qualche modo dal muro, sollevando l'ascia dalla schiena e assumento una posa da combattimento.

\subparagraph{Toku:} bene, io non ho intenzione di permettere a questa cosa di tirarci dentro il muro o qualsiasi altra cosa stia per fare. Mi ci avvento contro con i miei pugnali.

\subparagraph{Arbitro:} voi altri, che fate?

\subparagraph{Uthred:} io l'affronterò con l'ascia, cercando di farlo allontanare da Ezekiel e dal tedoforo.

\subparagraph{Ezekiel:} io Potenzierò l'attacco di Toku con Colpo Guidato, un mio Trucchetto.

\subparagraph{Arbitro:} okay, tirate per i danni.

\subparagraph{Toku:} {\em (tira 2d6 (due pugnali) + d12 (attacco Potenziato e usa il risultato più alto)} Ho fatto 5!

\subparagraph{Uthred:} {\em (tira d8 (Danno dell'arma) + d4 (dado bonus) e usa il risultato più alto)} E sono 6 danni!

\subparagraph{Arbitro:} {\em (sottrae 7 (6 + 1 per l'attaccante in più) danni e prende nota che l'opponente è ora a 0pf, con un avanzo di 3 Danno)} la spingi indietro con un calcio, facendole perdere l'equilibrio e tagliandole un fianco. {\em (tira un \save{FOR} vs Danno Critico, che riesce)} La copia ruggisce emettendo un rumore da disturbo elettrico, ma è ancora in piedi.

\subparagraph{Uthred:} c'è spazio per un solo Uthred qui!

\subparagraph{Arbitro:} la copia di Uthred lascia cadere a terra l'ascia e si protende in avanti cercando di afferrare Toku. Fammi un \save{DES}.

\subparagraph{Toku:} {\em (tira un \save{DES})} ehm\ldots ho fatto 20.

\subparagraph{Arbitro:} {\em (tra i lamenti al tavolo)} la creatura afferra Toku e prova a schiacciarlo contro le pareti. Una trama blu pulsante si forma sulla loro superficie. Un attimo dopo, i colori prendono la forma di Toku e la copia esce avanzando dal muro. Tocca a voi, ragazzi.

\subparagraph{Ezekiel:} mai avrei pensato di dover scegliere tra l'uccidere Toku o Uthred. Userò l'Incantesimo Tocco Gelido che ho conservato per distruggere la copia di Uthred.

\subparagraph{Uthred:} e se dopo di questo è ancora in piedi, proverò a tagliargli la testa!

\break

\subparagraph{Arbitro:} può fare un \save{FOR} per resistere all'effetto {\em (tira un \save{FOR})}, ma fallisce! Tira per vedere quanta FOR Perde la copia di Uthred.

\subparagraph{Ezekiel:} {\em (tira un d4 per quantizzare i Punti Persi di FOR, come indicato dall'Incantesimo)} quattro!

\subparagraph{Arbitro:} {\em (controlla i suoi appunti per vedere se ciò azzera la FOR della creatura)} è abbastanza per prosciugare l'energia di questa creatura. Il tocco fa svanire il colore dell'essere non appena questi cade a terra inerte e si spegne di colpo, completamente distrutto.

\subparagraph{Uthred:} sì!

\subparagraph{Arbitro:} Ezekiel, non dimenticare di prendere 2 danni per il lancio dell'Incantesimo. Poi, sappiate che avete fatto un botto di rumore in questa stanza.

{\em (fa di nascosto un Tiro per Incontri Casuali per vedere se un qualsiasi mostro nelle vicinanze possa aver notato il rumore. Un risultato di 1 indica che dovrebbe avvenire un incontro, così tira facendo riferimento alla tabella degli incontri ostili che ha preparato per questa zona)}

\subparagraph{Ezekiel:} qui le cose si mettono male.

\subparagraph{Arbitro:} sentite il rumore di qualcosa che sta scendendo per la scalinata. Ricordate quella creatura assurda, simile a un cavallo e con la pelle durissima tipo corteccia, che vi ha teso un'imboscata la scorsa sessione?

\subparagraph{Uthred:} certo, l'abbiamo spinta giù nella fossa e poi siamo fuggiti da veri eroi.

\subparagraph{Arbitro:} be', questo essere è pressoché identico, ma invece di avere la taglia di un cavallo, è talmento grosso che a stento riesce a stringersi per le scale. Le sue fauci appaiono abbastanza grandi da potervi inghiottire per intero e le sue zampe anteriori terminano con artigli di presa che si estendono per circa sei piedi. Inutile dire che sta scendendo le scale con voi nel mirino e non sembra amichevole. {\em (Fallisce un \save{VOL} per la copia di Toku, poiché la vista di questo essere è sufficiente a terrorizzarla)} La copia di Toku vede questa cosa e immediatamente striscia indietro nel muro, sparendo nelle tessere.

\subparagraph{Ezekiel:} a me, l'idea di essere ingoiato per intero non piace per niente. Che possibilità ci sono di corrergli tra le zampe?

\subparagraph{Arbitro:} è piuttosto stretto tra le scale. Se vuoi provarci, ci sarebbe sicuramente bisogno di un bel \save{DES}.

\subparagraph{Uthred:} il mostro più piccolo era impaurito dal fuoco o sbaglio? Magari potremmo inviare su il tedoforo per cercare di tenerlo a bada.

\subparagraph{Arbitro:} lui è palesemente esitante\ldots Deve superare un \save{VOL} per seguire un ordine suicida come questo. Non si sa mai, però, potrebbe funzionare!

\subparagraph{Toku:} superarlo correndo o provare a impaurirlo mi sembra inutilmente rischioso quando proprio qui abbiamo la perfetta via d'uscita!

\subparagraph{Uthred:} il pozzo? La creatura riuscirebbe a entrarci?

\subparagraph{Arbitro:} improbabile. Di sicuro è troppo grossa per riuscirci facilmente.

\subparagraph{Ezekiel:} per quanto possa sembrare un suicidio, potrebbe essere la nostra unica speranza. Posso tirare una moneta o qualcos'altro dentro il pozzo?

\subparagraph{Arbitro:} getti un mezzo Scellino giù nel pozzo e, qualche secondo dopo, senti distante uno \mbox{``Splash!''}.

\subparagraph{Toku:} acqua!

\subparagraph{Ezekiel:} nella più rosea delle previsioni\ldots Come facciamo a sapere che non si tratti di acido o roba simile? Penso che possiamo trovare un modo per distrarlo abbastanza a lungo da permetterci di scappare su per scalinata.

\subparagraph{Arbitro:} mentre formulate questo piano, la creatura è intanto riuscita a entrare con forza nella stanza, sfiorando la parete di tessere, che emettono ondulazioni blu.

\subparagraph{Uthred:} oh, cavolo! Non andrà a finire bene.

\subparagraph{Ezekiel:} ottimo! Nel buco!

\subparagraph{Toku:} fidatevi di me! Sarò anche il primo a saltarci dentro.

\subparagraph{Arbitro:} state per saltarci dentro adesso?

{\em (Tutto il gruppo annuisce riluttante)}

\subparagraph{Arbitro:} vi immergete nelle tenebre del pozzo, cadendo per qualche secondo prima di tuffarvi in quella che sembra acqua ghiacciata, abbastanza profonda da farvi cadere in sicurezza. La torcia del tedoforo si è spenta e riuscite a malapena a orientarvi nella fossa nera come la pece prima di provare una sensazione di formicolio su tutto il corpo. \saves{VOL} per tutti!

{\em (Al tavolo si leva un coro di lamenti)}
}

\end{document}
