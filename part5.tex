\documentclass[itdr]{subfiles}

\begin{document}
~\vspace{14ex}
\dimagepage{council2}

\chapter{Condurre il Gioco}
\label{ch:condurre_il_gioco}

\paragraph{L’Essenza di un Buon Arbitraggio}
\index{Arbitro}
Un buon arbitraggio sta nel fornire alle persone che giocano scelte interessanti da poter fare e nell’assicurarsi che tali scelte abbiano un impatto significativo sulla situazione corrente e sul prosieguo del gioco.

\vfill

\paragraph{Sapere Quando Tirare}
\index{Tiri}
Quando le altre persone partecipanti descrivono cosa vorrebbero far fare ai rispettivi personaggi, tu, da Arbitro, hai solitamente tre opzioni:
\begin{enumerate}
	\item È qualcosa che il personaggio può fare in sicurezza.
	\item Non è possibile. Chiedi, \mbox{magari} dando suggerimenti, che venga scelto un altro approccio. 
	\item Potrebbe essere fattibile, ma c’è un rischio. Vanno tirati i dadi.
\end{enumerate}

\vfill

\paragraph{Una Nota sul Rischio}
\index{Rischio}
In genere, se le altre persone partecipanti stanno correndo un rischio, tu, da Arbitro, dovresti farglielo presente: una partita deve riservare sorprese, ma le persone che giocano con te devono avere la sensazione che siano state le loro decisioni nel gioco a condurre verso un rischio foriero di brutte sorprese.

Per esempio, quando i personaggi incontrano un mostro o un pericolo che è molto probabile sia in grado di ucciderli sul colpo, l’Arbitro dovrebbe assicurarsi che ogni partecipante sia a conoscenza di questa eventualità. Se vogliono abbattere una porta a colpi d’ascia, dovrebbero sapere che molto probabilmente il rumore allerterà chiunque si trovi lì vicino. La valutazione del rischio rispetto alla possibile ricompensa è una parte importante del gioco e qualunque partecipante dovrebbe sempre avere ciò di cui ha bisogno per poter fare una scelta informata.

\vfill

\paragraph{Comprendere i Punteggi di Abilità}
\index{Punteggi di Abilità}
3: il minimo per un essere umano, limitazioni severe in questo campo.\\
10: nella media per un essere umano.\\
15: capacità umana eccellente.\\
20: il picco per un essere umano, la più eccezionale genialità, ecc.

\vfill

\paragraph{Comprendere i Tiri Salvezza}
\index{Tiri Salvezza}
Un TS va fatto quando ci si espone a un rischio.

\subparagraph{TS di FOR:} evitare conseguenze nocive attraverso \mbox{l’esercizio} di una forza fisica o la sopportazione di uno sforzo a carico del corpo.

\subparagraph{TS di DES:} evitare conseguenze nocive attraverso reazioni rapide, controllo del corpo nella sua interezza e~grazia.

\subparagraph{TS di VOL:} evitare conseguenze nocive attraverso la concentrazione, l'autodisciplina e il controllo della magia.

\vfill
\break

\index{Ordine di Marcia e Sequenza dei Turni in Combattimento}
\index{Combattimento}
\index{Turni}
\paragraph{Ordine di Marcia e Sequenza dei Turni in Combattimento}
L’ordine di marcia stabilisce qual è il primo personaggio a essere preso da una trappola, chi è colto da un’imboscata dalle retrovie, ecc. Da Arbitro, fa’ domande sulle azioni dei personaggi in combattimento seguendo lo stesso ordine. Poi, gli attacchi sono raggruppati e si procede ai tiri.

\vfill

\index{Attacchi}
\index{Danno}
\index{Danno!bonus}
\index{Incrementare/Ridurre i Dadi del Danno}
\index{Ridurre i Dadi del Danno|see {Incrementare/Ridurre i Dadi del Danno}}
\index{Perdita di Punti Abilità}
\paragraph{Comprendere il Danno}
\subparagraph{Incrementare/Ridurre i Dadi del Danno:} la taglia del dado cambia di uno nella gamma d4--d12.

\subparagraph{Dadi Bonus del Danno delle Armi:} tirali con il dado del Danno della tua arma. Se non è \mbox{specificata} una taglia per il dado, allora è la stessa del dado del Danno della tua arma.

\subparagraph{Notazione per gli Attacchi:} NdX significa tira N dadi a~X facce e prendi il singolo risultato migliore. Se il mostro può bersagliare più opponenti, i dadi del Danno possono essere ripartiti di conseguenza e tirati come attacchi distinti.

\subparagraph{Perdita di Punti Abilità:} i tiri per i Punti Abilità Persi non sono tiri per il Danno, quindi non sono influenzati dall’Armatura né richiedono Tiri Salvezza da Danno Critico, salvo quando diversamente indicato. 

\vfill

\paragraph{Quanti Danni?}
\index{Danno}
Il Danno dalla caduta di rocce, da esplosioni e da altri fonti al di fuori del combattimento ordinario rientra nella gamma d4--d12 e va contato separatamente rispetto agli attacchi del combattimento. 

Considera che effetti potrebbe avere su di una comune persona: una caduta che è probabile possa ferire un personaggio inesperto potrebbe infliggere d6 Danno, ma un’enorme roccia che ne schiaccerebbe la maggior parte potrebbe farne d12. 

\index{Veleno}
\subparagraph{Veleno:} il veleno potrebbe Compromettere gli attacchi, causare la Perdita di Punti Abilità o effetti come cecità, Svantaggio a certi Tiri Salvezza, ecc., ma in genere è efficace solo contro bersagli viventi. 

\vfill

\paragraph{Tiri Fortuna}
\index{Tiri Fortuna}
Da Arbitro, a volte potresti volere un elemento di aleatorietà senza ricorrere a un Tiro Salvezza, in particolare in situazioni dettate dalla fortuna o che non rientrano nella casistica dei tre Punteggi di Abilità. Quand'è così, tira un d6: un tiro basso favorisce i personaggi-giocanti e uno alto significa che sono stati colpiti dalla malasorte. Sta a te, poi, decidere cosa significhi in concreto un dato risultato rispetto alla situazione attuale.

\vfill

\paragraph{Tiri Conoscenza}
\index{Tiri Conoscenza}
I personaggi hanno 2 possibilità su 6 di sapere qualcosa al di fuori della loro area di conoscenza e delle loro esperienze passate; quelli specializzati hanno 4 possibilità su 6 relativamente al loro vasto campo di studi e conoscono ogni cosa che riguardi la loro ristretta specializzazione (p. es. Storia (Archeologia)).

\vfill
\end{document}
