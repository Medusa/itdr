\documentclass[itdr]{subfiles}

\begin{document}

\addtocontents{toc}{\protect\newpage}
\chapterx{Appendix A: Additional and Alternative Rules}
\label{ch:appendix_a}

``\title'' is intended as a rules-light game. Keep this in mind when deciding to use any of the rules presented in this appendix.
%\vspace{-0.4ex}
\vfill

\section{Characters}
\index{Characters}

\subsection{Balanced Characters}
Instead of the Extra Roll becoming your starting money, drop it. Your starting money is 21 minus the average of your Ability Scores (round up).

\subsection{Epic Characters}
If you want player characters to be more powerful, roll 2d6~+~6 for Ability Scores, and d4~+~2 for HP.

\subsection{Mundane Characters}
If you want player characters to be ordinary people, roll 2d8~+~1 for Ability Scores, and d6 for HP. \mbox{Take no Features but choose a Background as usual.}

To advance such character to Novice, choose a Feature, reroll HP taking the better result, and roll d20 for each Ability Score. If the roll is higher than the Ability Score, it increases by 1 (up to 18).

\subsection{Fortune Favours the Brave}
When creating the character or advancing to a new Experience Level, instead of choosing a new Feature, roll for a random one, including random Spells, \mbox{Expertise}, and Gifts\safepageref{ (see page }{random_features}{)}, to gain one of the following benefits afterwards:
\begin{itemize}
	\item Roll for HP once more and take the better result.
	\item Increase one Ability Score by 1 (up to 20).
\end{itemize}

\vfill

\section{Contest}
\index{Contest}
In an opposed contest when a simple Save would not suffice, both sides roll a Save. If one of the sides succeeds, it wins. If both sides succeed, the lower roll wins. On a tie, the higher Ability Score wins.

If weapons are involved, the attack might be subtracted from the Save or added to opponent's one.

\vfill

\section{Group Saves}
\index{Saves!group}
\index{Group Saves|see {Saves, group}}
When the whole group acts as one, a Group Save could be made. It is successful if more than half the characters pass their Saves. An attempt to steer a vessel in a storm could prompt a STR~Group Save, and sneaking past guards --- a DEX~Group save.

\vfill

\begin{comment}
\section{Gods, Religion, and Disciples}
\index{Religion}
\index{Disciple}
\index{Creed}

The nature of divine presence is highly dependent on a specific setting and thus is left to your discretion. Some worlds could be completely devoid of divine influence (though local cults might still have supernatural powers from some other source), while dwellers of other worlds can regularly observe their gods' interventions in the deals of mortals.

\subparagraph{Disciple} Class and its Creeds from the \textbf{\customref{ch:appendix_c}{Appendix C: Class-ic Edition}} could be used as a Feature to represent the most devoted adepts of cryptic cults. Unlike other Features, this one has a prerequisite of the character being a worshipper of the relevant set of teachings. When obtaining a new Experience Level, follow standard rules. Additionally, from Expert onwards, Disciples gain d4 (up to their WIL~/~2, rounded down) Followers (3hp, Simple Weapon) each time they visit a friendly settlement  and are responsible for their food, shelter, equipment, etc.
\end{comment}

\vfill

\section{Hardcore Mode}
To increase the difficulty, use the following rules:
\begin{itemize}
	\item Mystics use the \textbf{Random Spell Selection} rule.
	\item When mystic fail their \save{WIL} from casting a spell at 0hp, they suffer a \textbf{Magic Mishap}.
	\item Characters that take \textbf{Critical Damage} need an ally to spend their action to treat the wound or will lose d6~STR on each subsequent turn.
	\item Use the \textbf{Injuries} rule. \textbf{Broken limb} roll results in a lost limb instead. \textbf{Gravely injured} roll \mbox{results} in instant death.
\end{itemize}

\vfill

\section{Injuries}
\index{Injuries}
\index{Damage!critical}

On a failed Critical Damage Save, roll for an injury.
Effects of an injury could be fixed by Healing.

\begin{dtable}[cL]
	\textbf{d20} & \textbf{Injury} \\
	1--4	& \textbf{Bruise.} Nothing serious.\\
	5--7	& \textbf{Scar.} This will leave a mark.\\
	8--9	& \textbf{Concussion.} Disadv. on \saves{WIL}.\\
	10--11	& \textbf{Cracked rib.} Disadv. on \saves{DEX}.\\
	12--13	& \textbf{Torn muscle.} Disadv. on \saves{STR}.\\
	14--15	& \textbf{Broken gear.} Order: shield $\rightarrow$ armour $\rightarrow$ weapon. Fixing gear costs half its price.\\
	16		& \textbf{Fractured skull.} d6 WIL Loss.\\
	17		& \textbf{Broken ribs.} d6 DEX Loss.\\
	18		& \textbf{Internal bleeding.} d6 STR Loss.\\
	19		& \textbf{Broken limb.} Roll for a limb. Second hit to the same limb will result in its loss. (d4) \mbox{1--2:} left/right arm (cannot use it), \mbox{3--4:} left/right leg (cannot run, jump, etc.)\\
	20		& \textbf{Gravely injured.} Next failed Critical Damage Save will result in death.\\
\end{dtable}

\vfill

\section{Light}
\index{Light}

Torches, lanterns, and campfires illuminate in a \mbox{30-ft} radius. Big bonfires might cast light twice as far. Candles and such illuminate only in a 10-ft radius and thus are not commonly used by adventurers.

Mist, smoke, and such reduce the radius in half.

\subparagraph{Torch} lasts for about an hour. When used as a weapon, it deals d4 Fire Damage but might go out.

\subparagraph{Lantern} lasts for about four hours, can be dimmed at any moment, and refilled with lamp oil.

\vfill

\section{Living Expenses}
\index{Living Expenses}
\index{Lifestyle|see {Living Expenses}}

\equip{Squalid}{d4s/month}:
Suffer d4 Ability Score Loss for each Ability Score, your reputation suffers.

\equip{Adequate}{10$\times$d4s/month}:
Restore d4 Ability Score Loss for each Ability Score.

\equip{Luxury}{d4g/month}:
Heal any Ability Score Loss and non-magical ailments, your reputation rises.

If you own pets, add half as much for each one.

Halve the expense if you live in your own house.

\vfill

\section{Load Capacity}
\index{Load Capacity}

Characters can lift a maximum amount of load equal to their STR squared (in pounds). Half of this load can be carried without any impediment in speed. Twice as much can be dragged on the ground.

\begin{dtable}[llLl]
	\textbf{STR} & \textbf{Carry (\sfrac{1}{2}), lb} & \textbf{Lift, lb} & \textbf{Drag ($\times$2), lb} \\
	1	&	\sfrac{1}{2}		&	1		&	2		\\
	2	&	2		&	4		&	8		\\
	3	&	4\sfrac{1}{2}		&	9		&	18		\\
	4	&	8		&	16		&	32		\\
	5	&	12\sfrac{1}{2}		&	25		&	50		\\
	6	&	18		&	36		&	72		\\
	7	&	24\sfrac{1}{2}		&	49		&	98		\\
	8	&	32		&	64		&	128		\\
	9	&	40\sfrac{1}{2}		&	81		&	162		\\
	10	&	50		&	100		&	200		\\
	11	&	60\sfrac{1}{2}		&	121		&	242		\\
	12	&	72		&	144		&	288		\\
	13	&	84\sfrac{1}{2}		&	169		&	338		\\
	14	&	98		&	196		&	392		\\
	15	&	112\sfrac{1}{2}		&	225		&	450		\\
	16	&	128		&	256		&	512		\\
	17	&	144\sfrac{1}{2}		&	289		&	578		\\
	18	&	162		&	324		&	648		\\
	19	&	180\sfrac{1}{2}		&	361		&	722		\\
	20	&	200		&	400		&	800		\\
\end{dtable}

1~lb equals 100 gold guilders, 1000 silver shillings, or 1000 copper pennies in weight.

\subsection{Encumbrance}
\index{Encumbrance}
Aside from speed penalty, a heavy load reduces HP to 0. The same HP reduction happens when carrying more than three bulky items. Items are considered bulky if they require both hands to carry or otherwise unwieldy, for example, two-handed weapons, armour, a Mystic's Tome, a pot of black powder, etc.


\vfill
\break


\section{Madness}
\index{Madness}
\index{Sanity|see {Madness}}
\index{Insanity|see {Madness}}

If your game is heavily focused on a horror aspect, you might want to track characters' sanity.

Each time the character suffers an exposure to a source of supernatural dread, succeed on a \save{WIL} or gain a Madness Level.

A good night's sleep lowers Madness Level by 1.

When Madness Level exceeds character's Experience Level, the character goes insane. Roll for immediate and prolonged effect. Prolonged effects require a Healing Service to get rid of.

\begin{dtable}[cL]
	\textbf{d20} & \textbf{Immediate Effect} \\
	1--4 & \textbf{Shudder.} \\
	5--7 & \textbf{Scream} loudly, making a lot of noise. \\
	8--10 & \textbf{Flail} around, attacking a random nearby target on your next turn. \\
	11--13 & \textbf{Panic} and run away. 2-in-6 chance to drop your weapon while doing so. \\
	14--15 & \textbf{Frenzy.} Spend your turns attacking a random nearby target. After attacking an ally, succeed on a \save{WIL} to recover. \\
	16--17 & \textbf{Blindness} until Rest. \\
	18--19 & \textbf{Paralysis} until any incoming Damage, or someone takes an action to shake it off.\\
	20 & \textbf{Faint}. Need to be tended to by an ally and have a Rest to regain consciousness. \\
\end{dtable}

\begin{dtable}[cL]
	\textbf{d20} & \textbf{Prolonged Effect} \\
	1--4 & \textbf{Nightmares.} \\
	5--7 & \textbf{Hallucinations.} By Referee's discretion. \\
	8--10 & \textbf{Muteness.} Lose the ability to speak. \\
	11--13 & \textbf{Phobia.} Attacks against the cause of the phobia are Impaired. \\
	14--15 & \textbf{Paranoia.} Disadvantage on \saves{WIL}. \\
	16--17 & \textbf{Dizziness.} Disadvantage on \saves{DEX}. \\
	18--19 & \textbf{Weakness.} Disadvantage on \saves{STR}. \\
	20 & \textbf{Stupor.} Cannot take any actions. \\
\end{dtable}

\paragraph{Vestigial Effects}
Some especially shocking experiences might leave a permanent mark on the character's psyche, usually in a light form of some Prolonged Effect, obsession, compulsive behaviour, etc.


\vfill
\break


\section{Magic Mishaps}
\label{sec:magic_mishaps}
\index{Magic!mishaps}

When Mystics fail their \save{WIL} from casting a spell, they suffer a Mishap. Other dangerous interactions with magic (improper use of a magical device, spellcasting inside an anti-magic zone, destruction of a magic item, etc.) might lead to a Mishap as well.

\begin{dtable}[cL]
	\textbf{d100} & \textbf{Mishap} \\
	1--4	&	You exude a strong smell for a day. (d4) 1:~mint, 2:~garlic, 3:~vinegar, 4:~sulphur.	\\
	5--8	&	Your clothes' colour changes randomly.	\\
	9--12	&	Your clothes grow one size. Gain Disadvantage on \saves{DEX} until amended.	\\
	13--16	&	Your eye colour changes randomly.	\\
	17--20	&	Your eyes shed bright light for a day.	\\
	21--24	&	Your hair colour changes to a random one (new hair grows normal).	\\
	25--28	&	Your hair falls out.	\\
	29--32	&	Your hair grows to a yearly amount.	\\
	33--36	&	Your skin acquires a saturated shade of a random colour for d12 months.	\\
	37--40	&	Your skin is covered in a random growth for d12 months. (d4) 1:~fur, 2:~scales, 3:~feathers, 4:~spines.	\\
	41--43	&	You disappear for a minute.	\\
	44--46	&	You are stunned until Rest.	\\
	47--49	&	You fall unconscious until Rest.	\\
	50--52	&	You are invisible for an hour or until you attack or cast a spell.	\\
	53--55	&	Your ears become pointed and hairy.	\\
	56--58	&	You are deafened until Rest.	\\
	59--61	&	Your voice is very loud until Rest.	\\
	62--64	&	You are mute until Rest.	\\
	65--67	&	You see invisible things for an hour.	\\
	68--70	&	You are blinded until Rest.	\\
	71--72	&	You are obscured by a cloud of smoke.	\\
	73--74	&	Your HP drop to 0.	\\
	75--76	&	Your HP are restored.	\\
	77--78	&	You double in size for an hour. Gain Advantage on \saves{STR} and increase your weapon Damage dice by one (up to d12).	\\
	79--80	&	You halve in size for an hour. Gain Disadvantage on \saves{STR} and lower your weapon Damage dice by one (down to d4).	\\
	81--82	&	Your main weapon shrinks to one-sixteenth its size for an hour.	\\
	83--84	&	Your tongue becomes forked.	\\
	85--86	&	Your canines grow long and sharp.	\\
\end{dtable}
\begin{dtable}[cL]
\textbf{d100} & \textbf{Mishap} \\
	87--88	&	You grow a tail.	\\
	89--90	&	You grow horns.	\\
	91	&	You grow gills.	\\
	92	&	Your feet turn into hooves.	\\
	93	&	Your nails grow into sharp talons (bonus d6 unarmed Damage die).	\\
	94	&	Your skin becomes very tough. Gain Armour~1 when not wearing any armour.	\\
	95	&	One of your Ability Scores increases by one (up to 20). (d6) 1--2:~STR, 3--4:~DEX, 5--6:~WIL.	\\
	96	&	One of your Ability Scores decreases by one (down to 3). (d6) 1--2:~STR, 3--4:~DEX, 5--6:~WIL.	\\
	97	&	You grow a random body part.	\\
	98	&	You lose a random body part.	\\
	99	&	Your clothes burst in flames. Take d6 damage now and d6 at the end of your next turn, unless extinguished.	\\
	100	&	You are petrified.	\\
\end{dtable}

\vfill

\begin{dtable}[cLcL]
	\textbf{d12} & \textbf{Colour} & \textbf{d12} & \textbf{Colour} \\
	1 & snow white		& 7	& lemon yellow \\
	2 & ash grey		& 8	& malachite green \\
	3 & jet black		& 9 & sky blue \\
	4 & crimson red		& 10& ultramarine blue \\
	5 & chestnut brown	& 11& lavender violet \\
	6 & pumpkin orange	& 12& orchid magenta \\
\end{dtable}

\vfill

\begin{dtable}[cLcLcL]
	\textbf{d12} & \textbf{Body Part} & \textbf{d12} & \textbf{Body Part} \\
	1		& tooth	& 7--9 	 & toe	\\
	2--4	& finger& 10--11 & foot	\\
	5--6	& arm	& 12	 & eye	\\
\end{dtable}
\begin{comment}
% pre-1.3 version
\begin{dtable}[cLcL]
	\textbf{d20} & \textbf{Part} & \textbf{d20} & \textbf{Part} \\
	1		& tooth	& 11--14 & toe	\\
	2--5	& finger& 15--17 & foot	\\
	6--8	& hand	& 18--19 & leg	\\
	9--10	& arm	& 20	 & eye	\\
\end{dtable}
\end{comment}

\vfill

\section{Manufacture of Magic Equipment}
\label{sec:manufacture_of_magic_equipment}
\index{Magic!items}
\index{Focus}
\index{Scrolls}

Basic magic equipment can be created by a Mystic by spending the required amount of funding and time.

\subparagraph{Focus}: 10s in consumables, 1 day, a suitable item.

\subparagraph{Scroll}: 20s $\times$ Circle in consumables, 1 day $\times$ Circle. Successful on X-in-6, \mbox{X = 1 + Mystic Level -- Circle}, consumables are lost either way. (Designing a new spell, if the Referee allows it, costs and takes at least $\times$10 as much and requires some rare ingredients.)

\vfill

\section{Pets' Experience}
\index{Pets}
\index{Experience Levels!pets}

If you want to allow experience for pets, do it once, when a pet survives three Adventures. Use the same Ability Score and HP increase rules as characters do.

\break

\section{Rations}
\index{Rations}

While seafaring or travelling through inhospitable land it might be important to know the amount and weight of the rations required in your journey.

\begin{dtable}[Lllll]
	\textbf{Daily ration} & \textbf{Cost} & \textbf{Food} & \multicolumn{2}{l}{\textbf{Water}} \\
	Human	& 5p	& 2~lb	& \sfrac{1}{2}~gal	& (4~lb) \\
	Horse	& 1p	& 20~lb	& 5~gal	& (40~lb) \\
	Elephant& 1s	& 200~lb& 50~gal& (400~lb) \\
\end{dtable}

A day without enough water or a week without enough food results in d4 STR Loss.

\vfill

\section{Resources}
\index{Resources}

\begin{dtable}[cLcL]
	\textbf{Amount} & \textbf{Description} & \textbf{Average} & \textbf{Price} \\
	1 & running out	& 1		& $\times$~1 \\
	2 & low			& 2		& $\times$~d6 \\
	3 & enough		& 4		& $\times$~2d6 \\
	4 & plenty		& 7		& $\times$~3d6 \\
	5 & excess		& 13	& $\times$~4d6 \\
\end{dtable}

Each time you spend a resource (or after combat for ammo), roll a d6. If you roll over the Amount, decrease it by one. On zero the resource is depleted.

If you scavenge for the resource, roll a d6. If you roll over the Amount, increase it by 1 (up to 5).

When buying resources to increase the Amount by 1 (up to 5), pay its price multiplied by your current Amount~$\times$~d6.

\vfill

\section{Selling}
\index{Selling}
\index{Barter}

A chance to find a buyer for a pricey object is \mbox{X-in-6} based on a settlement and item's cost. You can repeat the search in the same settlement after d6 months.

\begin{dtable}[lLLLLLl]
	\textbf{gold:} &	\textbf{1+} & \textbf{10+} & \textbf{100+} & \textbf{1k+} & \textbf{10k+} & \textbf{100k+} \\
	Village	& 2	& 1		& ---	& ---	& ---	& --- \\
	Town	& 4 & 3		& 2		& 1		& ---	& --- \\
	City	& 6	& 5		& 4		& 3		& 2		& 1 \\
\end{dtable}

After finding a buyer, make a WIL Save. On a failed save, you sell for a \sfrac{1}{4} price. If you roll under your WIL Score by 10 or more, you sell for a full price, otherwise you sell for a \sfrac{1}{2} price. The chance of barter instead of a monetary exchange is (6--X)-in-6.

\subparagraph{Selling Magic Items} will have a higher chance of barter, while search roll and \save{WIL} are rolled at Disadvantage. Price for scrolls is d10s~$\times$~Circle, consumables: d10$\times$10s~$\times$~Circle, wands and rods: d10g~$\times$~Circle, other items --- on a case-by-case \mbox{basis.}

\vfill
\break

\section{Structures and Sieges}
\label{sec:structures_and_sieges}
\index{Structures}
\index{Sieges}

\subsection{Construction}
\index{Construction}
\index{Property}
\index{Walls}

\begin{dtable}[lLll]
	\multicolumn{2}{L}{\textbf{Structure}} & \textbf{Wood} & \textbf{Stone} \\
	\multicolumn{2}{L}{Bridge}							& 1g	& 5g \\
	\multicolumn{2}{L}{Building, 1 floor, P=120~ft}		& 1g	& 5g \\
	\multicolumn{2}{L}{Gatehouse, P=120~ft}				& 10g	& 50g \\
	\multicolumn{2}{L}{Keep, small, P=160~ft}			& 20g	& 100g \\
	\multicolumn{2}{L}{Keep, big, P=240~ft}				& ---	& 300g \\
	\multicolumn{2}{L}{Tower, small, P=80~ft}			& 5g	& 25g \\
	\multicolumn{2}{L}{Tower, big, P=120~ft}			& 10g	& 50g \\
	\multicolumn{2}{L}{Wall, 100~ft}					& 5g	& 25g \\
	\hline
	Dungeon, 10~ft cube & \multicolumn{3}{L}{1g (earth), 5g (rock)} \\
	Moat, 100~ft	& \multicolumn{3}{L}{1g (earth), 5g (rock)} \\
	Road, 1~mile	& \multicolumn{3}{L}{5g, $\times$2 on Rugged terrain} \\
\end{dtable}
{\em (P --- external perimeter of the building.)}

\subparagraph{Construction Crew} (four dozen people lead by a master, paid 50s per week) build 5g of structure cost weekly, 1g for stone construction. Up to 5 crews can work on a single structure simultaneously. Speed and cost might be impacted by external factors.

\subparagraph{Siege Engines} could be installed on gatehouses (1), big towers (1), small keeps (2), and big keeps (4).

\vfill

\subsection{Siege Engines}
\index{Siege Engines}
\index{Damage!siege engines}

Require a crew of three and a whole turn to reload. A reduced crew will reload in two or three turns.

\begin{dtable}[LlLL]
	\textbf{Engine} & \textbf{Cost} & \textbf{Damage} & \textbf{Ammo} \\
	Ballista 	& 1g 	& d12 		& 10s bolt \\
				& 	 	& d10 		& 5s ball  \\
	\rowcolor{dColor1}\multirow{-2}{*}{Catapult} & \multirow{-2}{*}{1g}	& d10 Blast & 20s bomb \\
	\rowcolor{dColor2}Cannon & 2g	& d12 Blast & 25s shot \\
\end{dtable}

The weight of a siege engine is about 1 ton and it requires a draft animal to transport overland.

\vfill

\subsection{Structural Damage}
\index{Damage!structural}

Armour range represents thickness of the material.

Large and bigger objects usually ignore damage from anything but siege engines and such.

\begin{dtable}[lcL|lc]
	\textbf{Size} & \textbf{HP} & \textbf{Example} & \textbf{Material} & \textbf{Armour} \\
	small	& 2--4	& chest	& ice 	& 2--4	\\
	medium	& 4--8	& wagon	& wood	& 4--6	\\
	large	& 6--12	& wall	& stone	& 6--8	\\
	huge	& 8--16	& ship	& metal	& 8--10	\\
\end{dtable}

For example, a small wooden ship will have 8hp and Armour~5 (wood of medium thickness).


\vfill
\break


\cleartoleftpage
\section{Travel}
\index{Travel}

Travel for 8 hours/day before resting for the night.

\subparagraph{Grid} of \travelunit{1}-mile tiles simplifies distance calculations.

\index{Terrain}
\begin{dtable}[lLcc]
\textbf{Terrain} & \textbf{Example} & \textbf{Miles} & \textbf{Grid} \\
Clear		& grassland, road, trail 			& \travelunit{5} & 5 \\
Rugged		& desert, forest, hills, mud, snow, river ford, mountain pass	& \travelunit{4} & 4 \\
Difficult 	& jungle, mountains, river, swamp	& \travelunit{3} & 3 \\
\end{dtable}

To speed up calculations, choose a dominant \mbox{terrain} for each day of travel and apply it to the whole day.

\vfill

\index{Large Groups}
\index{Forced March}
\begin{dtable}[Lcc]
	\textbf{Speed Modifier} & \textbf{Miles} & \textbf{Grid} \\
	
	\textbf{Concurrent activities}
	(exploring, sneaking, foraging, etc.) & --\travelunit{2} & --2 \\
	
	\textbf{Large groups} & --\travelunit{1} & --1 \\
	
	\textbf{Harsh weather} & --\travelunit{1} & --1 \\
	
	\textbf{Extreme weather} & --\travelunit{2} & --2 \\
	
	\multicolumn{3}{l}{\textbf{Mounts (except mules, donkeys, etc.):}} \\
	\hspace{0.5em}\labelitemi~Clear terrain & +\travelunit{1} & +1 \\
	\hspace{0.5em}\labelitemi~Camels in a desert & +\travelunit{1} & +1 \\
	\hspace{0.5em}\parbox{\linewidth}{\vspace{0.2ex}
	\labelitemi~Difficult terrain\\
	(except for elephants in a jungle)} & --\travelunit{1} & --1 \\
	
	\multicolumn{3}{l}{\textbf{Vehicles:}} \\
	\hspace{0.5em}\labelitemi~Clear terrain except roads & --\travelunit{1} & --1 \\
	\hspace{0.5em}\labelitemi~Rugged terrain & --\travelunit{2} & --2 \\
	\hspace{0.5em}\labelitemi~Difficult terrain & \multicolumn{2}{c}{impassable} \\
	
	\hline
	
	\multicolumn{3}{c}{\textbf{Multipliers} (applied last rounding up)} \\

	\textbf{Encumbrance} over 50~lb on foot or vehicle overload & \multicolumn{2}{c}{$\times$\sfrac{1}{2}} \\
	
	\textbf{Forced march} \mbox{for extra 4 hours,} \mbox{\save{STR} or lose d4 STR} \mbox{(if mounted --- for your mounts)} & \multicolumn{2}{c}{$\times$1\sfrac{1}{2}} \\
\end{dtable}

\index{Rest}
\subparagraph{Rest} for a day for every 6 days travelled or make Forced march \save{STR} against d4 STR Loss.

\vfill

\index{Mounts}
\index{Mule}
\index{Donkey}
\index{Horse}
\index{Camel}
\index{Elephant}
\begin{dtable}[Lccl]
	\textbf{Mount}  & \textbf{Cargo} & \textbf{Riders} & \textbf{Cost} \\
	Mule, Donkey	& \sfrac{1}{5}~t (400~lb)	& 1	& 20s \\
	Horse, Camel	& \sfrac{1}{4}~t (500~lb)	& 2 & 1g \\
	Elephant		& 2~t (4000~lb)	& 8	& 5g \\
\end{dtable}

\vfill

\index{Vehicles}
\index{Cart}
\index{Carriage}
\index{Wagon}
\begin{dtable}[Lcccl]
	\textbf{Vehicle} & \textbf{Horses} & \textbf{Cargo} & \textbf{Passengers} & \textbf{Cost} \\
	Cart	& 1	& \sfrac{1}{2}~t& 4	& 30s \\
	Carriage& 2 & 1~t			& 8	& 60s \\
	Wagon	& 4	& 2~t			& 16& 1g \\
\end{dtable}

\subparagraph{Passengers} occupy \sfrac{1}{8}~t of cargo space. Cargo and Riders/Passengers values are mutually exclusive.


\break


\subparagraph{Going Astray} is a possibility when traversing unfamiliar or heavily obscured terrain, in a dense fog or heavy rain, etc. If you have some additional advantage in navigation, roll a d6; otherwise, roll a d4:

\begin{dtable}[cL]
	\textbf{d6} & \textbf{Outcome} \\
	1	& \textbf{Lost!} You wander off to an unknown place. \\
	2	& \textbf{Going in circles.} No travel progress today. \\
	3	& \textbf{Meandering.} Halved travelled distance. \\
	4--6& \textbf{On course.} \\
\end{dtable}


\subparagraph{Horizon} is 3 miles away on a flat surface (for an Earth-sized planet), 12 miles at 100~ft height (ship's mast, tower), etc.: $dist.~(miles) \approx \sqrt{1.5 \times height~(ft)}$.

\vfill
\dimage{waterborne}{75pt}

\subsection{Waterborne Travel}
\index{Travel!waterborne}
\index{Waterborne Travel|see {Travel, waterborne}}
\index{Boats}
\index{Ships}

Travel for 12 hours/day. With a double crew, you can switch shifts to continue travelling at night.

\index{Vehicles!waterborne}
\index{Rowboat}
\index{Sailboat}
\index{Keelboat}
\index{Sailing Ship}
\index{Longship}
\index{Galley}
\begin{dtable}[lCCCCl]
\textbf{Vehicle} & \textbf{Miles} & \textbf{Grid} & \textbf{Crew} & \textbf{Cargo} & \textbf{Cost} \\
Raft (100~ft\textsuperscript{2})& \travelunit{2}&2 & 1& \sfrac{1}{4}~t & --- \\
Rowboat		& \travelunit{3}	& 3	& 1		& 1~t	& 50s \\
Sailboat	& \travelunit{12}	& 12& 1		& 5~t	& 15g \\
Keelboat	& \travelunit{6}	& 6	& 10	& 20~t	& 25g \\
Longship	& \travelunit{18}	& 18& 50	& 10~t	& 100g \\
Sailing ship& \travelunit{18}	& 18& 10	& 100~t	& 150g \\
Galley		& \travelunit{18}	& 18& 100	& 150~t	& 200g \\
\end{dtable}

Keelboats, longships and galleys have both sails and oars but cannot go against the wind under sail.

\subparagraph{Covered Distance} depends on weather and other conditions. Going upstream reduces covered distance by \travelunit{2} miles/day, and going downstream increases it by the same amount. Makeshift rafts move downstream only, with a speed of the stream.

\index{Fare}
\subparagraph{Fare} might vary from 1p per person to cross a river or lake to 1s per person for each 5 miles travelled in a long-distance voyage.

\subparagraph{Passengers} occupy 1~t of cargo space or half as much for short-distance travel.

\subparagraph{Rations} of food and water for one person take up \sfrac{1}{10}~t (200~lb) of cargo space per month of travel.

\index{Siege Engines}
\subparagraph{Siege Engines} could be mounted on keelboats (1), sailing ships (2), and galleys (3).


\break


\subsection{Weather}
\index{Weather}

Keep in mind that different climates might require adjusting the tables. For example, you might want to use the Sky table with a d8 or d12 roll for dry climates or d12~+~8 for rainy ones.

To decide for how many days the current weather persists, choose an appropriate die from d4 to d12, depending on the climate and weather type.

\vfill

\begin{dtable}[cL]
	\textbf{d6} & \textbf{Temperature} \\
	1	& colder than usual \\
	2--5& normal \\
	6	& warmer than usual \\
\end{dtable}

\vfill

\begin{dtable}[cL|cl]
	\textbf{d20} & \textbf{Sky} & \textbf{d20} & \textbf{Sky} \\
	1--4 & clear	& 13--14 & drizzle or fog \\
	5--8 & cloudy	& 15--18 & rain or snow \\
	9--12& overcast & 19--20 & storm or snowstorm \\
\end{dtable}

\vfill

\begin{dtable}[cL]
	\textbf{d8} & \textbf{Wind Direction} \\
	1--3& adverse \\
	4--5& side \\
	6--8& favourable \\
\end{dtable}
When following prevailing wind's direction, roll 2d8 and take the higher result; when going against it --- take the lower one.

\vfill

\subparagraph{Wind Force} might affect your sailing speed.

\begin{dtable}[cLcc]
	~ & \textbf{Wind} & \multicolumn{2}{c}{\textbf{Sailing Multiplier}} \\
	\textbf{d20} & \textbf{Force} & \textbf{Adverse or Side} & \textbf{Favourable} \\
	1--2	& calm		& $\times$0 & $\times$0 \\
	3--6	& breeze	& $\times$\sfrac{1}{3}	& $\times$\sfrac{1}{2} \\
	7--14	& average	& $\times$\sfrac{1}{2}	& $\times$1 \\
	15--18	& strong	& $\times$\sfrac{2}{3}	& $\times$1\sfrac{1}{2} \\
	19--20	& gale		& $\times$0	& $\times$2 \\
\end{dtable}

\vfill

Ships exposed to gale in open sea roll for a gale damage each 6 hours.

\begin{dtable}[cL]
	\textbf{d8} & \textbf{Gale Damage} \\
	1 	& \textbf{Wrecked.} Ship, cargo, and \sfrac{1}{2} crew is lost. \\
	2 	& \textbf{Broken mast.} No sailing speed. \\
	3 	& \textbf{Broken half of oars.} \sfrac{1}{2} rowing speed. \\
	4 	& \textbf{Torn sail.} \sfrac{1}{2} sailing speed. \\
	5--6& \textbf{Overboard.} Lost d6 crew members. \\
	7--8& \textbf{All is fine.} \\
\end{dtable}

\vfill

\begin{dbox}
	\subparagraph{Harsh weather} might impede vision, ranged combat and prohibit Resting before a shelter is found.
	
	\subparagraph{Extreme weather} (blizzard, hail, etc.) might even inflict continuous Damage (usually d4/hour).
\end{dbox}

\break

\subsection{Aerial Travel}
\index{Travel!aerial}
\index{Aerial Travel|see {Travel, aerial}}

Flying creatures travel for 8 hours/day before resting for the night. Flying magic items have energy to function for the same daily amount of time.

\index{Mounts}
\begin{dtable}[llCCC]
	\textbf{Mount} & \textbf{Example} & \textbf{Miles} & \textbf{Grid} & \textbf{Riders} \\
	Small				& pixie		& \travelunit{8}  & 8  &---\\
	Medium 				& harpy		& \travelunit{8}  & 8  & 1 \\
	Large				& griffon	& \travelunit{16} & 16 & 2 \\
	Large, fast			& pegasus	& \travelunit{24} & 24 & 2 \\
	Huge				& dragon	& \travelunit{16} & 16 & 8 \\
	Magic device	& broom		& \travelunit{16} & 16 & 2 \\
	Magic vehicle& carpet	& \travelunit{8}  & 8  & 8 \\
\end{dtable}

Full speed is only possible with \sfrac{1}{2}~of riders or less. Otherwise, the speed is halved.

\index{Vehicles!aerial}
\subparagraph{Aerial Vehicles} travel for 12 hours/day. Double crew allows to continue travelling at night.

\index{Balloon}
\index{Airship}
\begin{dtable}[lccccl]
	\textbf{Vehicle} & \textbf{Miles} & \textbf{Grid} & \textbf{Crew} & \textbf{Cargo} & \textbf{Cost} \\
	Balloon	& 40 & 8 & 1 & 1~t & 25g \\
	Airship	& 40 & 8 & 10& 10~t& 200g \\
\end{dtable}

Balloons and airships are affected by winds in the same manner as sailing ships.

\subparagraph{Balloons} always follow the wind direction. Each 3 hours of travel you may change altitude to catch a preferable wind (roll for a new wind direction).

\vfill

\subsection{Movement in Combat and Exploration}
\index{Combat!movement}
\index{Movement}
\index{Exploration}
\index{Turns}
Each \textbf{combat turn (1 minute)} characters move their travel Grid value $\times$ 10 feet (generally \textbf{40 feet}; $\pm$10~feet for clear or difficult terrain; $\times$\sfrac{1}{2} if encumbered).

For time-tracking purposes, exploration activities take \textbf{10 minutes}: searching, lockpicking, resting, etc.

\vfill

\begin{dbox}
	\index{Units of Measure}
	\index{Distance|see {Units of Measure}}
	\index{Volume|see {Units of Measure}}
	\index{Weight|see {Units of Measure}}
	\subsection*{Units of Measure}
	
	\subparagraph{Distance}
	\begin{itemize}
		\item \textbf{1 mile} is 1760 yards or 5280 feet
		\item \textbf{1 yard} is 3 feet or 36 inches
		\item \textbf{1 foot} is 12 inches
	\end{itemize}

	\subparagraph{Volume}
	\begin{itemize}
		\item \textbf{1 gallon} is 4 quarts or 8 pints
		\item \textbf{1 quart} is 2 pints or 32 ounces
		\item \textbf{1 pint} is 16 ounces
	\end{itemize}
	
	\subparagraph{Weight}
	\begin{itemize}
		\item \textbf{1 ton} is 2000 pounds
		\item \textbf{1 pound} is 16 ounces
		\item \textbf{1 pound} is 100 gold guilders
		\item \textbf{1 pound} is 1000 silver shillings
		\item \textbf{1 pound} is 1000 copper pennies
	\end{itemize}
\end{dbox}

\break

\end{document}
