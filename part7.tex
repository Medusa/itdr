\documentclass[itdr]{subfiles}

\begin{document}

\chapter{Insidie e Ostacoli}
\label{ch:Insidie_e_ostacoli}
\index{Insidie}
\index{Trappole}
\index{Ostacoli}
\index{Esplorazione}

\paragraph{Riconoscere le Insidie}
Come regola di massima, la presenza di una trappola o~di un'altra insidia è sempre notata dai personaggi a~meno che essi non stiano correndo, la loro vista non sia ridotta o non siano distratti. Detto ciò, i~personaggi potrebbero essere in seguito danneggiati a causa di un mancato intervento o di mancanza di cautela. Giocatrici e giocatori dovrebbero considerare modi creativi per aggirare un'insidia o per disarmarla completamente. Sistemi rischiosi potrebbero richiedere un Tiro Salvezza o un Tiro Fortuna.

\vfill
\paragraph{Porte Chiuse}
\index{Porte}
In genere, con un po' di tempo a disposizione una porta chiusa può essere aperta con un grimaldello. Non occorre alcun Tiro Salvezza a meno che non ci sia il rischio di innescare una trappola, allertare forze nemiche o finire il tempo a propria disposizione.

I tentativi di usare di fretta grimaldelli e altra attrezzatura mentre si è sotto pressione richiedono di solito un \save{DES} e potrebbero includere la necessità di accendere una torcia mentre si è sotto attacco o di legare una corda prima che una persona amica precipiti verso la morte.

Buttare giù una porta è fattibile allo stesso modo senza un Tiro Salvezza, a meno che non ci siano rischi o~pressione, il che potrebbe imporre un \save{FOR}. Tuttavia, abbattere una porta causa sempre molto rumore e~può richiedere parecchio tempo.

\vfill
\paragraph{Incontri Casuali}
\index{Incontri}
\index{Incontri Casuali|see {Incontri}}
Qualsiasi cosa possa muoversi in un dungeon è improbabile che resti tutto il tempo nello stesso posto. Pertanto, l'Arbitro dovrebbe considerare che ci sia una probabilità che un gruppo incontri mostri o quant'altro. Fare rumori forti aumenta o diminuisce le possibilità che questo accada, a seconda della natura dell'incontro.

Da Arbitro, quando i personaggi esplorano, Riposano o lanciano Incantesimi non preparati, oppure indugiano in un luogo pericoloso, tira un d6.

\begin{dtable}[cL]
	\textbf{d6} & \textbf{Esito} \\
	1	& Tira per un Incontro Casuale.\\
	2	& Tira per un Incontro Casuale. Dà un segnale che le creature sono vicine o sono passate di qui.\\
	3--6& Libero.\\
\end{dtable}

Attardarsi quanto serve per fare un pasto, o per dormire, impone di tirare invece un d4.
Dadi più grandi (dal d8 al d12) potrebbero essere usati per ambienti meno pericolosi.

\vfill
\break

\section{Esempi di Incontri Casuali}
\index{Incontri}

\header{Incontri nei Dungeon}
\begin{dtable}[cL]
	\textbf{2d4} & \textbf{Incontro} \\
	2	&	cubo gelatinoso	\\
	3	&	d4 rugginofagi	\\
	4	&	d8 scheletri	\\
	5	&	2d6 goblin	\\
	6	&	d6 orchetti	\\
	7	&	ingozza lerciume	\\
	8	&	orrore uncinato	\\
\end{dtable}

\vfill

\header{Incontri negli Ambienti Selvaggi}
\begin{dtable}[cL]
	\textbf{d4+d6} & \textbf{Incontro} \\
	2	&	orco	\\
	3	&	cavallo imbizzarrito	\\
	4	&	2d6 goblin, 2 su 6 possibilità di imboscata	\\
	5	&	d6 cacciatori	\\
	6	&	branco di 3d4 lupi	\\
	7	&	cinghiale selvatico	\\
	8	&	branco di 3d6 lupi	\\
	9	&	d4 cervi	\\
	10	&	orso	\\
\end{dtable}

\vfill

Le tabelle per gli incontri casuali possono essere usate anche in ambienti non ostili.

\header{Incontri Urbani}
\begin{dtable}[cL]
	\textbf{2d8} & \textbf{Incontro} \\
	2	&	rissa per strada; 2 possibilità su 6 che le~guardie cittadine siano già presenti	\\
	3	&	impertinente giovane che vive per strada e~che prova a rubare un borsello o un qualche oggetto a caso da un personaggio	\\
	4	&	servitù in gruppo che trasporta un palanchino	\\
	5	&	ambulante che vende merci esotiche	\\
	6	&	individuo ubriaco in cerca di guai	\\
	7	&	\mbox{individuo che promuove ad alta voce} \mbox{un vicino locale}	\\
	8	&	\mbox{mendicante claudicante all'angolo} \mbox{di una strada}	\\
	9	&	commerciante di cibo da strada	\\
	10	&	carretto rotto che blocca la strada	\\
	11	&	\mbox{pattuglia di guardia composta} \mbox{da 2d4 sentinelle}	\\
	12	&	gruppo di artisti di strada	\\
	13	&	\mbox{esponente del clero che raccoglie} \mbox{elemosine per un tempio locale}	\\
	14	&	\mbox{guardie che scortano un individuo} \mbox{arrestato per furto}	\\
	15	&	sfilata per una festività del posto	\\
	16	&	agguato (2d4 criminali) in un vicolo buio	\\
\end{dtable}

\vfill
\break

\section{Esempi di Trappole}
\index{Trappole}


\paragraph{Cerchio del Tradimento}
Innescata entrando in un cerchio contrassegnato da un simbolo che raffigura un pugnale conficcato in un cuore.

\save{VOL} o si attacca il bersaglio alleato più vicino e si continua a farlo finché non si prende la condizione Svenuto. Se si supera il Tiro Salvezza, si esce forzatamente dal cerchio e si prende d6 Danno.

\vfill
\paragraph{Gabbia a Fossa}
Una botola è visibile a meno che il personaggio non sia distratto, non stia correndo o la sua vista non sia ridotta. Si attiva calpestando la botola.

Innescare la trappola infligge d8 Danno. Una gabbia di metallo intrappola la vittima finché non viene liberata con una chiave e un allarme mobilita qualche creatura sgradevole.

\vfill
\paragraph{Rampicanti Afferranti}
Si innesca quando ci si avvicina a rampicanti \mbox{dall'aspetto} strano. Si prende d6 Danno ogni \mbox{turno} finché non ce ne si libera. \save{FOR} ogni turno per liberarsi, altrimenti si resta immobili. Altamente infiammabili.

\vfill
\paragraph{Sporgenza da Percorrere Tenendosi in Equilibrio}
Deve essere percorsa per raggiungere qualunque cosa si~trovi dall'altra parte. Può essere fatto in sicurezza senza pressione addosso, ma se c'è da correre o si è sotto attacco, bisogna superare un \save{DES} o si cade al livello inferiore e per arrampicarsi su occorrerà una corda.

Il livello inferiore ospita dei coccodrilli (FOR~13, DES~5, VOL~5, 9pf, Armatura~1, d8~Morso).

\vfill
\paragraph{Trappola a Dardo Narcotizzante}
Alla base di un forziere è visibile una cerbottana, che viene innescata aprendo il forziere senza prendere le appropriate precauzioni. Dardi rotti sono disseminati sul pavimento di questa stanza.
d8 Danni. d8 Punti di DES Persi in caso di Danno Critico.

\vfill
\paragraph{Trappola a Lama Oscillante}
Oscilla senza sosta sopra un corridoio seguendo una sequenza. Può essere bloccata solo da un'asta di metallo molto forte o da altri oggetti appropriati.

\save{DES} per superarla senza farsi del male, altrimenti si prende d10 Danno mentre si passa.

\vfill
\break

\section{Esempi di Ostacoli}
\index{Ostacoli}

\paragraph{Anomalia Gravitazionale}
Una zona con forza di gravità alterata (direzione o~intensità).

\vfill
\paragraph{Attivazione Remota}
Una saracinesca che si apre girando una ruota nella stanza accanto.

\vfill
\paragraph{Barriera Mentale}
Un muro di forza che blocca esclusivamente gli esseri senzienti coscienti.

\vfill
\paragraph{Dimensioni Distorte}
Il dungeon non segue le leggi comuni della geometria poiché esiste in un insieme diverso di dimensioni.

\vfill
\paragraph{Fortezza Volante}
Un'antica struttura che, seguendo un percorso giornaliero, fluttua a un'altezza inarrivabile, passando talvolta molto vicino alla locale catena montuosa.

\vfill
\paragraph{Galleria Incompiuta}
C'è una grotta sconosciuta dietro appena un paio di piedi di roccia. Rumori o altri segnali potrebbero suggerire la sua presenza.

\vfill
\paragraph{Passaggio Sott'Acqua}
Una stanza allagata con un tunnel sul fondo.

\vfill
\paragraph{Pavimento di Cristallo}
Un pavimento fatto di un materiale cristallino più liscio del ghiaccio. Il movimento è molto difficile e il rischio di cadere e di scivolare lungo un pendio è onnippresente.

\vfill
\paragraph{Sala Controllo}
Una stanza piena di leve e di bottoni che scambiano corridoi, cancelli e dispositivi nascosti attraverso il~dungeon. Non sono presenti segni o istruzioni.

\vfill
\paragraph{Sfera di Negazione della Magia}
Un misterisoso dispositivo, situato in cima a una colossale guglia di pietra, che risucchia l'energia magica e~disattiva così Incantesimi e oggettistica magica quanto più ci si avvicina a esso. Si inizia con gli Incantesimi del 5° Cerchio e si finisce lasciando i Mistici con i~loro soli Trucchetti nelle sue più immediate vicinanze. Anche gli oggetti magici hanno una chance ridotta di funzionare correttamente (seguendo la stessa falsariga, da~5~possibilità su 6 a 1 su 6).

\vfill

\end{document}
