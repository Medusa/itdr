\documentclass[itdr]{subfiles}

\begin{document}

\begin{minipage}{\textwidth}
\toc[2] % Table of Contents
\end{minipage}

\vfill

\begin{minipage}{\textwidth}
\begin{center}\footnotesize
	Scrittura e illustrazioni di \author{}. Basato su ``Into the Dungeon: Playtest Edition'' di Chris McDowall.
	
	Revisione di Galina Arabadzhi. Playtester: Elthari, Kailes, LordOfMemes, Shadko, Thalia, Veyalitsa, WolfyTechnoChan et al.
\end{center}
\end{minipage}

\clearpage

\chapterx{Introduzione}
\label{ch:introduction}

\dimagebottom{dice}{72pt}

\section*{Il Mondo}
\index{Esplorazione}

Il mondo è talmente grande che non c’è chi sia in grado di rappresentarlo su di una mappa e talmente vecchio che i testi scritti non possono documentarlo del tutto. Le città crescono dalle rovine di civiltà cadute prima di esse e nuova tecnologia fiorisce accanto ad antichi congegni. \mbox{Giramondo} fanno ritorno dall'Avventura da ogni direzione con racconti di luoghi bizzarri, sia meravigliosi che orribili.

Siete Avventuriere e Avventurieri che sfidano l’ignoto in cerca di ricchezza, fama, conoscenza o potere.

\section*{Le Divinità}
\index{Divinità}

La loro immagine è dipinta dalle tribù sulle pareti delle grotte e ogni angolo del mondo ha il suo folclore. I bellicosi uomini di Baru, la città rubata, venerano i quattro fratelli e il cereo pellegrinaggio anela il ritorno dal vuoto profondo del suo padre perduto. C’è chi sostiene le divinità ci abbiano donato sapere e magia e chi invece obietta che i Mistici le facciano infuriare a causa dei loro oscuri studi.

\section*{Mostri}
\index{Mostri}

La Saggia Baizin spese la sua breve vita a compilare un bestiario della fauna conosciuta, ma sapeva che il cercare in luoghi troppi oscuri, profondi o distanti le avrebbe mostrato cose troppo terribili per poter essere riportate su di una pagina. Gli sciocchi individui che se ne vanno in giro in cerca di mostri da uccidere di solito trovano la morte in breve tempo.

\section*{Rune e Magia}
\index{Runico}
\index{Magia}

Chi sa decifrare le Rune arcane accede al sapere perduto della storia umana e dei riti segreti che rilasciano Incantesimi di grande potere. I Mistici credono che esso sia un lascito delle divinità morte e studiano gelosamente i loro Tomi mentre cercano Pergamene che accrescano la loro conoscenza e incrementino il loro potere.

\vfill
\break

~\vspace{-17.5pt}
\section*{Sopravvivere al Mondo}

Chi va all’Avventura potrebbe possedere capacità eccezionali o avere accesso a magia di grande potere, ma non esiste chi sia in grado di sopravvivere a una gola tagliata o a una caduta in una fossa profonda cento piedi. Correte, sgattaiolate, arrendetevi o corrompete: qualunque cosa vi consenta di ottenere ciò che vi serve e di aver salva la vita è semplicemente efficace quanto il combattere.

\section*{Andare Oltre le Spedizioni}

C’è un lungo elenco di audaci Avventuriere e Avventurieri che hanno trovato la morte: solo pochi individui vivono così a lungo da poter passare a imprese più grandi in veste di generali, capi di un culto o imperatori. Ci sono perfino storie di chi ha scoperto il vero potere delle divinità ed ha raggiunto lo status divino.

\vspace{3.5pt}
\dimage{intro2}{120pt}

\section*{A Digiuno di GdR?}

\index{Arbitro}
\subsection*{Come si gioca?}
Una persona gioca in veste di Arbitro e si occupa di descrivere la situazione in cui si trovano i personaggi delle altre persone che partecipano al gioco. Quest’ultime potrebbero fare domande e far interagire con l'ambiente i rispettivi personaggi. L’Arbitro dirà loro cosa accade e se serve che tirino i dadi per stabilire l’esito delle loro azioni.

\subsection*{Cosa occorre per giocare?}
Un set di dadi poliedrici, matite e carta. L’Arbitro prepara un luogo da far esplorare ai personaggi o usa un modulo d’Avventura già pronto.

\end{document}
