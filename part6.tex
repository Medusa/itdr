\documentclass[itdr]{subfiles}

\begin{document}

\chapter{Tesori e Magie}
\label{ch:tesori_e_magie}
\index{Tesoro}
\index{Denaro}
\index{Magia}

\paragraph{Ricchezze}
In base al tipo, dalle gemme alle opere d'arte agli oggetti funzionali, ogni tesoro ha un certo valore. Commercianti e acquirenti spesso desiderano trattare sul prezzo o, in caso di oggetti cha valgono migliaia di Scellini, potrebbero non poterselo permettere affatto.

\vfill

\paragraph{Monete}
Tutte le monete sono valutate rispetto allo \textbf{Scellino d'Argento Standard} (s). Con uno Scellino puoi permetterti un pasto, una bevanda e un letto decente per la notte e corrisponde alla paga guadagnata da una persona comune per una settimana tipo di lavoro.

C'è un'enorme varietà di monete che è valutata rispetto allo Scellino, con due tipologie che sono particolarmente comuni.

\subparagraph{Pennies di Rame} (p): valgono un decimo di Scellino. Con un Penny puoi permetterti una bevanda a buon mercato in una pessima taverna o un passaggio su di un traghetto.

\subparagraph{Fiorini d'Oro} (f): valgono cento Scellini. Con un Fiorino puoi permetterti un buon cavallo, un'armatura completa o un gioiello prezioso.

\vfill

\paragraph{Creazione di Nuovi Incantesimi}
\index{Incantesimi}
Quando vi cimentate nella creazione di nuovi Incantesimi, usate il \textbf{\fullref{ch:magia}} come riferimento per livelli di potere e possibili effetti.

\index{Incantesimi!danno}
Stima approssimativa del Danno:
\begin{itemize}
	\item \textbf{Trucchetti}: d4
	\item \textbf{1° Cerchio}: da d4 a d6
	\item \textbf{2° Cerchio}: da d6 a d8
	\item \textbf{3° Cerchio}: da d8 a d10
	\item \textbf{4° Cerchio}: da d10 a d12
	\item \textbf{5° Cerchio}: d12
\end{itemize}

Incantesimi a effetto continuato e ad area infliggono di solito meno Danno rispetto a quelli instantanei dello stesso Cerchio.

\index{danno!elementale}
\index{Danno Elementale|see {Danno, elementale}}
Alcuni Incantesimi potrebbero infliggere Danno \mbox{elementale:} i più comuni sono Elettricità, Freddo e~Fuoco.

Tiri salvezza appropriati contro determinati effetti:
\begin{itemize}
	\item \textbf{FOR:} ostacali fisici, Incantesimi a contatto, metamorfosi e altri effetti che influenzano il corpo
	\item \textbf{DES:} elusione, equilibrio, estinzione delle fiamme
	\item \textbf{VOL:} paura, illusioni e controllo mentale.
\end{itemize}

\vfill

\paragraph{Rompere le Regole}
Non tutta la magia funziona come qulle dei Mistici. La~magia può fare qualunque cosa e non ha limiti.

\vspace{10ex}
\break

\paragraph{Armi e Armature Magiche}
\index{Armi Magiche|see {Magia, oggetti}}
\index{Armi}
\index{Armature|see {Magia, oggetti}}
\index{Armature}
\index{Runico}
Le armi create con il potere della magia hanno spesso incise dei simboli in Runico che ci dicono il loro nome, la loro storia e il loro scopo. Oltre ad avere un \textbf{dado del Danno Incrementato} (fino a d10) e a \textbf{ignorare le resistenze soprannaturali}, le armi magiche avranno anche una \textbf{proprietà extra}, come il venire avvolte dalle fiamme quando fanno sanguinare o il guidare in direzione dell'oro. Non si tratterà mai di \mbox{fare} semplicemente più Danni, sebbene alcune armi potrebbero causare \mbox{\textbf{effetti aggiuntivi}} quando provocano Danno Critico, come il traformare la vittima in pietra.

Similmente, le armature magiche e gli scudi magici avranno una \textbf{proprietà extra} oppure offriranno una \textbf{maggiore protezione} contra una specifica fonte di~Danno.

\vfill

\paragraph{Altri Oggetti Magici}
\index{Magia!oggetti}
Gli altri oggetti magici potrebbero includere anelli, mantelli, guanti e pendagli. Essi potrebbero offrire a chi li indossa un \textbf{effetto continuato} o richiedere un'\textbf{attivazione}. L'effetto di solito non sarà esattamente lo stesso di un Incantesimo, ma potrebbe essere simile.

\vfill

\index{Consumabili|see {Magia, oggetti}}
\subparagraph{Oggetti Magici Consumabili:} le pozioni e gli oggetti magici similari offriranno a chi li consuma un beneficio una tantum.

\vfill

\index{Anelli Magici|see {Magia, oggetti}}
\subparagraph{Anelli Magici:} sono limitati a non più di uno per mano.

\vfill

\index{Bacchette|see {Magia, oggetti}}
\index{Verghe|see {Magia, oggetti}}
\subparagraph{Bacchette e Verghe:} hanno un numero limitato e sconosciuto di cariche. Dopo il primo uso, tira un d4 e~annota il risultato. Ogni volta che usi l'oggetto, tira un d6: se fai pìù di quel numero, allora diminuiscilo di uno. A zero, l'oggetto viene prosciugato e diventa inutilizzabile.

\vfill

\paragraph{Inconvenienti e Maledizioni}
In genere, la maggior parte degli oggetti magici più potenti riserva a chi l'utilizza un qualche tipo di inconveniente o detrimento, \mbox{permanente} oppure che si verifica a ogni utilizzo. Tali proprietà non possono essere rivelate grazie a un Incantesimo \textit{Identifica}, ma unicamente attraverso la sperimentazione e l'uso.

\vfill
\dimage{treasures}{112pt}
\vfill
\break


\section{Esempi di Oggetti Magici}
\index{Magia!oggetti}

\paragraph{Amuleto di Protezione della Salute}
Quando viene trovato, questo amuleto di rubino ha un Potere di 2d6+6.

Qualunque Danno al Punteggio di~FOR è invece sottratto al Potere dell'amuleto, poi si tira un d20: se il tiro è più alto del Potere dell'amuleto, per quel giorno non lo si potrà usare di nuovo. Quando il Potere arriva a 0, l'amuleto va in pezzi.

\vfill
\paragraph{Anello della Rigenerazione}
Questo anello di salice fa recuperare 1~Punto di FOR Perso al giorno.

\vfill
\paragraph{Diadema dell'Empatia}
Un sottile diadema di vetro che consente a chi lo indossa di percepire i veri sentimenti e le emozioni altrui.

\vfill
\paragraph{Elmo della Respirazione}
Se necessario, questo elmo di cristallo fornisce a chi lo indossa fino a un'ora di approviggionamento di aria pulita.

\vfill
\paragraph{Guanti di Seta di Ragno}
Fatti di seta di ragno incantata, questi eleganti guanti consentono a chi li indossa di arrampicarsi su qualsiasi superficie. Questa stessa proprietà adesiva potrebbe anche imporre Vantaggio o Svantaggio ai Tiri Salvezza appropriati.

\vfill
\paragraph{Mantello della Discesa}
Questo mantello di cuoio rallenta la velocità di caduta di chi lo utilizza, consentendo anche di muoversi e~scivolare per un breve tratto.

\vfill
\paragraph{Maschera del Camuffamento}
Questa raffinata maschera d'argento consente a chi la usa di assumere l'aspetto esteriore di un \mbox{viso} altrui una volta al giorno.

\vfill
\paragraph{Mutapelle}
Questa pelle di animale trasforma in una creatura corrispondente chi la indossa. Ogni volta che un personaggio la indossa, va tirato un d100: con un 1, la mutapelle non può essere tolta finché la maledizione non viene rimossa. La probabilità aumenta di un 1\% per ogni successivo uso da parte dello stesso personaggio.

\vfill
\paragraph{Scopa Volante}
Questa scopa può trasportare fino a due \mbox{esseri umani} che le sono saliti in sella. Può essere usata anche come Focus del Mistico.

\vfill
\paragraph{Tappeto Volante}
Questo tappeto dal disegno particolare è leggero come una piuma e può trasportare in aria fino a 8 esseri umani (ma solo alla metà della velocità di una scopa volante).

\vfill
\break

\subsection*{Armature e Armi}

\paragraph{Armatura di Legno Ferro}
Qualsiasi Danno da Elettricitià non può ignorare questa \mbox{armatura} completa fatta con un legno scuro innaturalmente forte.

\paragraph{Bastone del Cobra}
Una delle estremità di questo bastone ricurvo (d8, 2-mani) è una testa di cobra stilizzata. Assieme al Danno, fa anche Perdere~d4~Punti di DES (soggetti all'Armatura).

\paragraph{Boomerang Fortunato}
Questo esotico boomerang d'avorio trova sempre il suo bersaglio e per questo annulla Compromissioni da copertura o simili.

\paragraph{Scudo a Specchio}
Questo scudo d'acciao lucidato a specchio ha delle possibilità di riuscire a bloccare un Incantesimo in arrivo in base al relativo Cerchio: \mbox{0--1 = 3 su 6, 2--3 = 2 su 6, 4--5 = 1 su 6}. Un Incantesimo bloccato a 2 su 6 possibilità di ritorcersi contro chi lo ha lanciato.

\vfill

\subsection*{Consumabili}

\paragraph{Ago della Negazione}
Quando questo sottile ago d'argento viene rotto, interrompe per un minuto gli effetti in corso degli Incantesimi in un'area piccola.

\paragraph{Quadrifoglio}
Ripeti un singolo Tiro Salvezza fallito, poi il quadrifoglio appassisce.

\paragraph{Pozione Curativa}
Questa fiala di liquido rosso frizzante fa recuperare d6 Punti di FOR Persi.

\index{Veleno}
\paragraph{Veleno Mortale}
Se consumato, questo scuro liquido olioso fa Perdere d6~Punti di FOR e costringe a un Tiro Salvezza da Danno Critico. Se si fallisce questo TS, si muore. Se applicato su di un'arma idonea o su di una serie di munizioni, i tiri per il Danno Critico dovuto a esse vanno fatti con Svantaggio finché non ci si Riposa.

\vfill

\subsection*{Bacchette e Verghe}

\paragraph{Bacchetta della Scossa}
Questa bacchetta d'ambra infligge d6 Danno da Elettricità che \mbox{ignora} l'Armatura.

\paragraph{Verga della Rivelazione}
Questa verga d'ossidiana rivela illusioni, entità Invisibili, porte segrete, trappole, ecc., nella direzione verso cui è rivolto.

\vfill

\begin{dbox}
	Vedi \textbf{\safenameref{subsec:oggetti_magici_random}{Oggetti Magici Random}} nell'\textbf{\customref{ch:appendice_a}{Appendice A}} per \mbox{ulteriore ispirazione}.
\end{dbox}

\vfill

\end{document}
